\documentclass{article}

\usepackage{tabularx}
\usepackage{booktabs}

\title{Problem Statement and Goals\\\progname}

\author{Aaron Billones\\Gillian Ford\\Juan Moncada\\Steven Ramundi}

\date{}

\input{../Comments}
\input{../Common}

\begin{document}

\maketitle

\begin{table}[hp]
\caption{Revision History} \label{TblRevisionHistory}
\begin{tabularx}{\textwidth}{llX}
\toprule
\textbf{Date} & \textbf{Developer(s)} & \textbf{Change}\\
\midrule
Date1 & Name(s) & Description of changes\\
Date2 & Name(s) & Description of changes\\
... & ... & ...\\
\bottomrule
\end{tabularx}
\end{table}

\section{Problem Statement}

\wss{You should check your problem statement with the
\href{https://github.com/smiths/capTemplate/blob/main/docs/Checklists/ProbState-Checklist.pdf}
{problem statement checklist}.}
\wss{You can change the section headings, as long as you include the required information.}

\subsection{Problem}

\subsection{Inputs and Outputs}

\wss{Characterize the problem in terms of ``high level'' inputs and outputs.  
Use abstraction so that you can avoid details.}

\subsection{Stakeholders}

\subsection{Environment}

\wss{Hardware and software}

\section{Goals}
\body{The main goal of this project is to develop a system for Sheridan Nurseries
that will replace the need for human workers in the current process of populating plant trays with pots. This will reduce the cost
of labour for the nursery, while being significantly less expensive than current alternatives. The goal of the system is to populate 
a standard tray with 10 pots in 30 seconds, equating to 2 trays per minute and 960 trays in an 8 hour shift. This is based on output 
numbers provided by the manager for the current system. Another goal for the project is to reload the trays and pots every 15 minutes.
This would require space for 30 trays and 300 pots.
}

\section{Stretch Goals}
\body{The first stretch goal is to double the potential output of the machine. This will approximatley equate the maximum intake of the
machine responsible for populating pots with soil. The second stretch goal is to increase machine capacity to 60 trays and 600 pots to
maintain the goal of reloading the machine every 15 minutes. The third stretch goal is to make the machine capable of dealing with variable
tray and pot sizes.}

\end{document}