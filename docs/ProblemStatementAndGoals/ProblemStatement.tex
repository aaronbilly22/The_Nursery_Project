\documentclass{article}

\usepackage{tabularx}
\usepackage{booktabs}

\title{Problem Statement and Goals\\The Nursery Project}

\author{Aaron Billones, billonea\\Gillian Ford, fordg\\Juan Moncada, moncadaj\\Steven Ramundi, ramundis}

\date{September 26. 2022}

%% Comments

\usepackage{color}

\newif\ifcomments\commentstrue %displays comments
%\newif\ifcomments\commentsfalse %so that comments do not display

\ifcomments
\newcommand{\authornote}[3]{\textcolor{#1}{[#3 ---#2]}}
\newcommand{\todo}[1]{\textcolor{red}{[TODO: #1]}}
\else
\newcommand{\authornote}[3]{}
\newcommand{\todo}[1]{}
\fi

\newcommand{\wss}[1]{\authornote{blue}{SS}{#1}} 
\newcommand{\plt}[1]{\authornote{magenta}{TPLT}{#1}} %For explanation of the template
\newcommand{\an}[1]{\authornote{cyan}{Author}{#1}}

%% Common Parts

\newcommand{\progname}{ProgName} % PUT YOUR PROGRAM NAME HERE
\newcommand{\authname}{Team \#, Team Name
\\ Student 1 name
\\ Student 2 name
\\ Student 3 name
\\ Student 4 name} % AUTHOR NAMES                  

\usepackage{hyperref}
    \hypersetup{colorlinks=true, linkcolor=blue, citecolor=blue, filecolor=blue,
                urlcolor=blue, unicode=false}
    \urlstyle{same}
                                


\begin{document}

\maketitle

\begin{table}[hp]
\caption{Revision History} \label{TblRevisionHistory}
\begin{tabularx}{\textwidth}{llX}
\toprule
\textbf{Date} & \textbf{Developer(s)} & \textbf{Change}\\
\midrule
2022-09-25 & Juan Moncada,\\&Aaron Billones,\\&Steven Ramundi,\\&Gillian Ford & Initial release\\
 
\bottomrule
\end{tabularx}
\end{table}

\newpage

\section{Problem Statement}

Sheridan Nurseries is one of Canadas largest nursery opperations and is growing year by year.
In the past few years, the company has taken steps to automate their production line and streamline their operation.
A significant amount of manual labour was seen in between the delivery of skidded pots/trays and the beginning of the propagation 
assembly line. In the current state, one employee is used to recieve stacks of pots and trays from their respective skids,
populate the trays with pots, and feed the now filled trays into the assembly line where the propagation of plants into the pots begins.
This process currently requires full-time labour from at minimum one employee, if not more, to yield the required output.

\subsection{Problem}
Sheridan Nurseries currently has no automation in their process of populating their trays and pots thus needing significant manual labour.

\subsection{Inputs and Outputs}
\subsubsection{physical input}
physcal inputs will be stacks of trays and pots.
\subsubsection{software input}
verifaction of trays and pots loaded. On commannd
\subsubsection{physical output}
filled trays with pots.
\subsubsection{software output}
series of warnings. load trays, load pots, verification failed.


\subsection{Stakeholders}
The main stake holders for this project will be the nursery manager, owner, aswell as the individual working the assembly line.

\subsection{Environment}
This project will be an even split between hardware and software as there is a need to build and incorporate into an existing assembly line.
Hardware will have to be used in order manipulate pots and trays, while software will be used to control said hardware 
aswel as take care of any verification that trays have been populated properly.
\section{Goals}
The main goal of this project is to develop a system for Sheridan Nurseries
that will replace the need for human workers in the current process of populating plant trays with pots. This will reduce the cost
of labour for the nursery, while being significantly less expensive than current alternatives. The goal of the system is to populate 
a standard tray with 10 pots in 30 seconds, equating to 2 trays per minute and 960 trays in an 8 hour shift. This is based on output 
numbers provided by the manager for the current system. Another goal for the project is to reload the trays and pots every 15 minutes.
This would require space for 30 trays and 300 pots.


\section{Stretch Goals}
The first stretch goal is to double the potential output of the machine. This will approximatley equate the output rate to maximum rate of the
machine responsible for populating pots with soil. The second stretch goal is to increase machine capacity to 60 trays and 600 pots to
maintain the goal of reloading the machine every 15 minutes. The third stretch goal is to make the machine capable of dealing with variable
tray and pot sizes.

\end{document}