\documentclass[12pt, titlepage]{article}

\usepackage{booktabs}
\usepackage{tabularx}
\usepackage{longtable}
\usepackage{hyperref}
\hypersetup{
    colorlinks,
    citecolor=black,
    filecolor=black,
    linkcolor=red,
    urlcolor=blue
}
\usepackage[round]{natbib}

%% Comments

\usepackage{color}

\newif\ifcomments\commentstrue %displays comments
%\newif\ifcomments\commentsfalse %so that comments do not display

\ifcomments
\newcommand{\authornote}[3]{\textcolor{#1}{[#3 ---#2]}}
\newcommand{\todo}[1]{\textcolor{red}{[TODO: #1]}}
\else
\newcommand{\authornote}[3]{}
\newcommand{\todo}[1]{}
\fi

\newcommand{\wss}[1]{\authornote{blue}{SS}{#1}} 
\newcommand{\plt}[1]{\authornote{magenta}{TPLT}{#1}} %For explanation of the template
\newcommand{\an}[1]{\authornote{cyan}{Author}{#1}}

%% Common Parts

\newcommand{\progname}{ProgName} % PUT YOUR PROGRAM NAME HERE
\newcommand{\authname}{Team \#, Team Name
\\ Student 1 name
\\ Student 2 name
\\ Student 3 name
\\ Student 4 name} % AUTHOR NAMES                  

\usepackage{hyperref}
    \hypersetup{colorlinks=true, linkcolor=blue, citecolor=blue, filecolor=blue,
                urlcolor=blue, unicode=false}
    \urlstyle{same}
                                


\begin{document}

\title{Verification and Validation Report: The Nursery Project} 
\author{Aaron Billones, billonea\\Gillian Ford, fordg\\Juan Moncada, moncadaj\\Steven Ramundi, ramundis}
\date{March 8, 2023}
	
\maketitle

\pagenumbering{roman}

\section{Revision History}

\begin{tabularx}{\textwidth}{p{3cm}p{2cm}X}
\toprule {\bf Date} & {\bf Version} & {\bf Notes}\\
\midrule
2023-03-08 & 1.0 & Initial Revision\\

\bottomrule
\end{tabularx}

~\newpage

\section{Symbols, Abbreviations and Acronyms}

\renewcommand{\arraystretch}{1.2}
\begin{tabular}{l l} 
  \toprule		
  \textbf{symbol} & \textbf{description}\\
  \midrule 
  ART & Accessibility Requirements Test\\
CR & Conveyor Functional Requirement\\
CST & Conveyor Subsystem Test\\
EPET & Expected Physical Environment Test\\
LCD & Liquid-Crystal Display\\
LED & Light Emitting Diode\\
LRT & Learning Requirements Test\\
MG & Module Guide\\
MIS & Management Information Systems\\
NFR & Non-Functional Requirement\\
PDST & Pot Dispenser Subsystem Test\\
PDR & Pot Dispensing Functional Requirement\\
PCST & Pot-pulator Complete System Testing\\
PT & Precision Test\\
RT & Reliability Test\\
SCT & Safety Critical Test\\
SRS & Software Requirements Specification\\
SRT & Speed Requirements Test\\
TDST & Tray Dispenser Subsystem Test\\
TDR & Tray Dispensing Functional Requirement\\
VST & Verification Subsystem Test\\
VR & Verification Functional Requirement\\
  \bottomrule
\end{tabular}\\

\newpage

\tableofcontents

\listoftables %if appropriate

\listoffigures %if appropriate

\newpage

\pagenumbering{arabic}

\section{Functional Requirements Evaluation}

\subsection{Pot-pulator Complete System Testing}

\begin{longtable}{ |p{0.12\textwidth}|*{4}{>{\centering\arraybackslash}p{0.20\textwidth}|}p{0.08\textwidth}|   }
  \caption{PCST Evaluation}
  \label{tab:PCST}\\
    \hline
    Test Number & Description & Input & Expected Output & Actual Output & Result\\
    \hline
    PCST-01 & Tray Dispenser Operation & Sensor reading of status of tray stack & Normal system operation & Normal system operation & Pass \\
    \hline
    PCST-02 & On Switch for Tray Dispenser Error & Pot-pulator switch set to on & Normal system operation & Normal system operation & Pass\\
    \hline
    PCST-03 & Pot Dispenser Operation & Sensor reading of status of pot stack & Normal system operation & Normal system operation & Pass\\
    \hline
    PCST-04 & On Switch for Pot Dispenser Error & Pot-pulator switch set to on & Normal system operation & Normal system operation & Pass\\
    \hline
    PCST-05 & Conveyor Operation & Tray placed on conveyor &  Normal system operation & Normal system operation & Pass\\
    \hline
    PCST-06 & Conveyor On Button & Pot-pulator switch set to on & Normal system operation & Normal system operation & Pass\\
    \hline
\end{longtable}

\subsection{Tray Dispenser Subsystem Testing}

\begin{longtable}{ |p{0.12\textwidth}|*{4}{>{\centering\arraybackslash}p{0.20\textwidth}|}p{0.08\textwidth}|  }
  \caption{TDST Evaluation}
  \label{tab:TDST}\\
    \hline
    Test Number & Description & Input & Expected Output & Actual Output & Result\\
    \hline
    TDST-01 & Tray Stack Detection & Sensor reads status of tray stack & Signal sent to microprocessor indicating trays are/are not present & Signal detected indicating presence of trays, no signal detected when trays are not present & Pass\\
    \hline
    TDST-02 & Operation from Tray Stack Detection & Sensor reads status of tray stack & Normal system operation & Normal system operation & Pass\\
    \hline
    TDST-03 & Tray from Stack to Conveyor & Stack of trays on tray dropper & One tray from stack is placed onto conveyor, next tray ready to be placed onto conveyor & One tray from stack is placed onto conveyor. Occasionally, two trays will be placed if there are only two remaining in the tray stack & Fail\\
    \hline
    TDST-04 & Verify Tray Status on Conveyor & Sensor reads status of tray on conveyor & Normal system operation & Tray dropper drops additional tray once first tray is no longer underneath tray dropper & Pass\\
    \hline
    
\end{longtable}

\subsection{Pot Dispenser Subsystem Testing}

\begin{longtable}{ |p{0.12\textwidth}|*{4}{>{\centering\arraybackslash}p{0.20\textwidth}|}p{0.08\textwidth}|  }
  \caption{PDST Evaluation}
  \label{tab:PDST}\\
    \hline
    Test Number & Description & Input & Expected Output & Actual Output & Result\\
    \hline
    PDST-01 & Pot from Stack to Tray & Simulated sensor input, two pot locations of tray directly below pot dispenser & Pot dispenser dispenses two pots into designated pot locations & Pot dispenser dispenses two pots, 70\% of cases tested saw pots dispensed into pot locations & Fail\\
    \hline
    PDST-02 & Tray Sensing & Trays placed in front of sensor & Signal sent to microprocessor indicating trays are/are not present & Signal detected indicating presence of trays, no signal detected when trays are not present & Pass\\
    \hline
    PDST-03 & Ability to Dispense 4" Diameter Pots & N/A & N/A & N/A & Pass\\
    \hline
    PDST-04 & Ability to Store/Dispense Multiple Pots & Ten pots, simulated sensor input & Pot dispenser dispenses two pots, reloads from stack, dispenses two pots, etc. until pot storage is empty & Pot dispenser dispenses two pots at a time for 5 cycles until all 10 pots are dispensed & Pass\\
    \hline
    PDST-05 & Pot Storage Detection & N/A & Signal output when no pots are detected in pot storage & Signal output when no pots are detected in pot storage & Pass\\
    \hline
    
\end{longtable}

\newpage

\subsection{Conveyor Subsystem Testing}

\begin{longtable}{ |p{0.12\textwidth}|*{4}{>{\centering\arraybackslash}p{0.20\textwidth}|}p{0.08\textwidth}|  }
  \caption{CST Evaluation}
  \label{tab:CST}\\
    \hline
    Test Number & Description & Input & Expected Output & Actual Output & Result\\
    \hline
    CST-01 & Conveyor Ability to Move Trays & Simulated inputs indicating conveyor can start & Constant speed of conveyor motor and belt & Constant speed of conveyor motor and belt & Pass\\
    \hline
    CST-02 & Conveyor Ability to Stop & Simulated signals from pot dispenser indicating tray is beneath pot dispenser & Conveyor motor and belt come to a stop & Conveyor motor and belt come to a stop & Pass\\
    \hline
    CST-03 & Conveyor Belt Friction & Mass of tray, tilt angle of conveyor belt & Maximum acceleration of conveyor belt & Maximum acceleration of conveyor belt & Pass\\
    \hline
    
\end{longtable}

\subsection{Verification Subsystem Testing}

\begin{longtable}{ |p{0.12\textwidth}|*{4}{>{\centering\arraybackslash}p{0.20\textwidth}|}p{0.08\textwidth}|  }
  \caption{VST Evaluation}
  \label{tab:VST}\\
    \hline
    Test Number & Description & Input & Expected Output & Actual Output & Result\\
    \hline
    VST-01 & Verify Correct Number of Pots in Tray & Tray filled with pots and tray not filled with pots & Signal sent indicating tray has not been completely filled with pots, no signal sent indicating all pot locations are filled & Signal sent indicating tray has not been completely filled with pots, no signal sent indicating all pot locations are filled & Pass\\
    \hline
    
    
\end{longtable}


\section{Nonfunctional Requirements Evaluation}

\subsection{Safety Critical Testing}

\begin{longtable}{ |p{0.12\textwidth}|*{4}{>{\centering\arraybackslash}p{0.20\textwidth}|}p{0.08\textwidth}|  }
  \caption{SCT Evaluation}
  \label{tab:SCT}\\
    \hline
    Test Number & Description & Input & Expected Output & Actual Output & Result\\
    \hline
    SCT-01 & Tray Dispenser Failure & Tray dispenser disconnect & System flags tray dispenser failure & System flags tray dispenser failure & Pass\\
    \hline
    SCT-02 & Pot Dispenser Failure & Pot dispenser disconnect & System flags pot dispenser failure & System flags pot dispenser failure & Pass\\
    \hline
    SCT-03 & Conveyor Failure & Conveyor disconnect & System flags conveyor failure & System flags conveyor failure & Pass\\
    \hline
    SCT-04 & Verification Failure & Verification disconnect & System flags verification failure & System flags verification failure & Pass\\
    \hline
    
    
\end{longtable}
		
\subsection{Precision Testing}

\begin{longtable}{ |p{0.12\textwidth}|*{4}{>{\centering\arraybackslash}p{0.20\textwidth}|}p{0.08\textwidth}|  }
  \caption{PT Evaluation}
  \label{tab:PT}\\
    \hline
    Test Number & Description & Input & Expected Output & Actual Output & Result\\
    \hline
    PT-01 & Tray Dispenser Precision & N/A & N/A & N/A & Pass\\
    \hline
    PT-02 & Pot Dispenser Precision & N/A & N/A & N/A & Pass\\
    \hline
    
\end{longtable}

\subsection{Reliability Testing}

\begin{longtable}{ |p{0.12\textwidth}|*{4}{>{\centering\arraybackslash}p{0.20\textwidth}|}p{0.08\textwidth}|  }
  \caption{RT Evaluation}
  \label{tab:RT}\\
    \hline
    Test Number & Description & Input & Expected Output & Actual Output & Result\\
    \hline
    RT-01 & Function Under Vibration & N/A & N/A & N/A & Pass\\
    \hline
    
\end{longtable}

\subsection{Expected Physical Environment Testing}

\begin{longtable}{ |p{0.12\textwidth}|*{4}{>{\centering\arraybackslash}p{0.20\textwidth}|}p{0.08\textwidth}|  }
  \caption{EPET Evaluation}
  \label{tab:EPET}\\
    \hline
    Test Number & Description & Input & Expected Output & Actual Output & Result\\
    \hline
    EPET-01 & Function Under Aerial Pollution & N/A & N/A & N/A & Pass\\
    \hline
    
\end{longtable}

\subsection{Speed Requirements Testing}

\begin{longtable}{ |p{0.12\textwidth}|*{4}{>{\centering\arraybackslash}p{0.20\textwidth}|}p{0.08\textwidth}|  }
  \caption{SRT Evaluation}
  \label{tab:SRT}\\
    \hline
    Test Number & Description & Input & Expected Output & Actual Output & Result\\
    \hline
    SRT-01 & Acceleration Displacement of Trays & N/A & N/A & N/A & Pass\\
    \hline
    SRT-02 & Pot Dispensing Rate & Stack of pots & Pots dispensed at desired rate & Pots dispensed at desired rate & Pass\\
    \hline
    SRT-03 & Tray Dispensing Rate & Stack of trays & Trays dispensed at desired rate & Trays dispensed at desired rate & Pass\\
    \hline

\end{longtable}

\subsection{Learning Requirements Testing}

\begin{longtable}{ |p{0.12\textwidth}|*{4}{>{\centering\arraybackslash}p{0.20\textwidth}|}p{0.08\textwidth}|  }
  \caption{LRT Evaluation}
  \label{tab:LRT}\\
    \hline
    Test Number & Description & Input & Expected Output & Actual Output & Result\\
    \hline
    LRT-01 & Operational Simplicity & N/A & N/A & N/A & Pass\\
    \hline

\end{longtable}

\subsection{Accessibility Testing}

\begin{longtable}{ |p{0.12\textwidth}|*{4}{>{\centering\arraybackslash}p{0.20\textwidth}|}p{0.08\textwidth}|  }
  \caption{ART Evaluation}
  \label{tab:ART}\\
    \hline
    Test Number & Description & Input & Expected Output & Actual Output & Result\\
    \hline
    ART-01 & Audio and Visual Indicators & Trigger signal & Corresponding light, sound, or screen display & Corresponding light, sound, or screen display & Pass\\
    \hline

\end{longtable}
	
\section{Comparison to Existing Implementation}	

After doing some research, there were some existing products that perform a similar 
function as the Pot-pulator. However, these products are large industrial machines that cost tens of thousands of dollars,
which is out of the price range for Sheridan Nurseries. Our goal was to develop a product that could perform the
the desired function of filling trays with pots at a inexpensive price point. The existing products are more robust and can operate
with larger loads within a given time frame. The Pot-pulator is intended to help solve Sheridan Nurseries' problem in an inexpensive
manner. We believe that we have achived this within a budget of only 750 dollars.

\section{Unit Testing}

  \subsection{Hardware Testing}
  The following section includes testing that was performed on critical hardware components
  of the system. These tests were performed for the purpose of verifying and understanding the  
  behaviour of each hardware component.
  \begin{enumerate}
  \item{HT-01: \textbf{AC Motor Control}}
  
  Test Description: The conveyor operates using an AC motor. In order to control it,
  a potentiometer and mechanical relay were used. The potentiometer controls the motor speed,
  and the relay controls on/off rotation of the motor.

  Input: Signal from Arduino to periodically switch the relay. Potentiometer position varies.

  Expected Output: Conveyor starts and stops based on the switching of relay. The speed of conveyor slows and speeds up
  in accordance to the potentiometer.

  Test Result: PASS
  \\
  \item{HT-02: \textbf{Stepper Motor Control}}\\
  Test Description: Many subsystems of the Pot-pulator uses stepper motors (ie. tray dispenser,pot dropper).
   Stepper drivers were wired and tested with step sizes that were offered by 
   the drivers purchased. 

  Input: Signals from Arduino to turn the motor 1 full revolution in full, half, quarter,
  eighth, and sixteenth stepping modes.

  Expected Output: 1 full revolution

  Test Result: PASS
  \\
  \item{HT-03: \textbf{Ultrasonic Range Finder Control}}\\
  Test Description: These sensors are used to meaure distances within the system for complete
  operation. Measurements were observed and verfied with a physical measuring tool (ie. ruler, tape measure)

  Input: Object placed in front of the sensor.

  Expected Output: Distance measured and displayed in the serial monitor (console).

  Test Result: PASS
\end{enumerate}

\section{Changes Due to Testing}

The changes due to testing are summarized below:
\\
\begin{enumerate}
  \item \textbf{Verification:}\\
  The bases of the verification mounts were recently expanded away from the
  conveyer by approximately 5 cm on either side, to address issues with the pots being too close to the 
  sensor when travelling down the conveyer, resulting in inaccurate readings.

  \item \textbf{Pot Dropper:}\\
  The part design of the threaded discs that drop the pots needed to be modified.
  After the inital testing of the pot dropper, the design resulted in pots being dropped
  in groups larger than 1. During each rotation of the stepper motors that are intended to drop 1 pot
  at a time, often times more than 1 pot would be dropped. Because of these results, the part design was 
  changed and tested. The part design underwent multiple design changes in order to have the highest success rate 
  when dropping pots.

  \item \textbf{Tray Dispenser:}\\
  The part design of the toothed cylinder which is responsible for dispensing the trays on to the conveyor
  was changed during testing. The diameter and number of teeth dictate whether or not the trays will drop one at a time with 
  each motor increment. Several models were designed and tested until a design with diameter of approximately 5cm with 12 teeth was found to be the 
  most effective model.
\end{enumerate}

\section{Automated Testing}
		
\section{Trace to Requirements}

The following table outlines all of the system tests and how they relate to the
relevent requirements. The requirements can be referenced in the SRS document.\\

\begin{longtable}{ |p{4cm}|p{8cm}|  }
  \caption{Corresponding Test IDs and Requirements}
  \label{tab:Table1}\\
  
  \hline
  \textbf{Test ID} & \textbf{Supporting Requirements}\\
  \hline
  TDST-01 &  TDR3, TDR5\\
  \hline
  TDST-02 &  TDR4, TDR5 \\
  \hline
  TDST-03 &  TDR2 \\
  \hline
  TDST-04 &  TDR2 \\
  \hline
  PDST-01 &  PDR2 \\
  \hline
  PDST-02 &  PDR2 \\
  \hline
  PDST-03 &  PDR3 \\
  \hline
  PDST-04 &  PDR4 \\
  \hline
  PDST-05 &  PDR5, PDR6 \\
  \hline
  PCST-01 &  TDR1 \\
  \hline
  PCST-02 &  TDR5, TDR6 \\
  \hline
  PCST-03 &  TDR7 \\
  \hline
  PCST-04 &  PDR1 \\
  \hline
  PCST-05 &  PDR6, PDR7 \\
  \hline
  PCST-06 &  PDR8 \\
  \hline
  VST-01 &  VR1 \\
  \hline
  VST-02&  VR2 \\
  \hline
  SCT-01 &  NFR12 \\
  \hline
  SCT-02 &  NFR12 \\
  \hline
  SCT-03 &  NFR12 \\
  \hline
  SCT-04 &  NFR12 \\
  \hline
  PT-01 & NFR13 \\
  \hline
  PT-02 & NFR14 \\
  \hline
  RT-01 & NFR17 \\
  \hline
  EPET-01 & NFR20 \\
  \hline
  LRT-01&  NFR6 \\
  \hline
  ART-01 & NFR7 \\
  \hline
  SRT-01 & NFR8 \\
  \hline
  SRT-02 & NFR9 \\
  \hline
  SRT-03 & NFR10 \\
  \hline
\end{longtable}

\newpage

		
\section{Trace to Modules}	

The following table outlines all of the system tests and how they relate to the
relevent modules. The modules can be referenced in the MG document.\\

\begin{longtable}{ |p{4cm}|p{8cm}|  }
  \caption{Corresponding Test IDs and Modules}
  \label{tab:Table1}\\
  
  \hline
  \textbf{Test ID} & \textbf{Supporting Modules}\\
  \hline
  TDST-01 &  M9\\
  \hline
  TDST-02 &  M9 \\
  \hline
  TDST-03 &  M10 \\
  \hline
  TDST-04 &   M10\\
  \hline
  PDST-01 &  M4 \\
  \hline
  PDST-02 &  M4 \\
  \hline
  PDST-03 &  M5 \\
  \hline
  PDST-04 &  M3 \\
  \hline
  PDST-05 & M6 \\
  \hline
  PCST-01 &  M9 \\
  \hline
  PCST-02 & M12 \\
  \hline
  PCST-03 &  M6 \\
  \hline
  PCST-04 &  M3 \\
  \hline
  PCST-05 & M3 \\
  \hline
  PCST-06 &  M7 \\
  \hline
  VST-01 &  M13 \\
  \hline
  VST-02&  M14 \\
  \hline
  SCT-01 &  M12 \\
  \hline
  SCT-02 &  M6 \\
  \hline
  SCT-03 &  M8 \\
  \hline
  SCT-04 &  M14 \\
  \hline
  PT-01 & M12 \\
  \hline
  PT-02 & M6 \\
  \hline
  SRT-01 & M12 \\
  \hline
  SRT-02 & M6 \\
  \hline
  SRT-03 & M12 \\
  \hline
\end{longtable}

\newpage

\section{Code Coverage Metrics}

\bibliographystyle{plainnat}
\bibliography{../../refs/References}

\newpage{}
\section*{Appendix --- Reflection}

The information in this section will be used to evaluate the team members on the
graduate attribute of Lifelong Learning.  Please answer the following questions:
\begin{enumerate}
  \item
In what ways was the Verification and Validation (VnV) Plan different
from the activities that were actually conducted for VnV? If there were
differences, what changes required the modification in the plan? Why
did these changes occur? Would you be able to anticipate these changes
in future projects? If there weren’t any differences, how was your team
able to clearly predict a feasible amount of effort and the right tasks
needed to build the evidence that demonstrates the required quality?
(It is expected that most teams will have had to deviate from their
original VnV Plan.)

\end{enumerate}

\end{document}