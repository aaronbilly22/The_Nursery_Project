\documentclass[12pt, titlepage]{article}

\usepackage{booktabs}
\usepackage{tabularx}
\usepackage{hyperref}
\hypersetup{
    colorlinks,
    citecolor=black,
    filecolor=black,
    linkcolor=red,
    urlcolor=blue
}
\usepackage[round]{natbib}

%% Comments

\usepackage{color}

\newif\ifcomments\commentstrue %displays comments
%\newif\ifcomments\commentsfalse %so that comments do not display

\ifcomments
\newcommand{\authornote}[3]{\textcolor{#1}{[#3 ---#2]}}
\newcommand{\todo}[1]{\textcolor{red}{[TODO: #1]}}
\else
\newcommand{\authornote}[3]{}
\newcommand{\todo}[1]{}
\fi

\newcommand{\wss}[1]{\authornote{blue}{SS}{#1}} 
\newcommand{\plt}[1]{\authornote{magenta}{TPLT}{#1}} %For explanation of the template
\newcommand{\an}[1]{\authornote{cyan}{Author}{#1}}

%% Common Parts

\newcommand{\progname}{ProgName} % PUT YOUR PROGRAM NAME HERE
\newcommand{\authname}{Team \#, Team Name
\\ Student 1 name
\\ Student 2 name
\\ Student 3 name
\\ Student 4 name} % AUTHOR NAMES                  

\usepackage{hyperref}
    \hypersetup{colorlinks=true, linkcolor=blue, citecolor=blue, filecolor=blue,
                urlcolor=blue, unicode=false}
    \urlstyle{same}
                                


\begin{document}

\title{Verification and Validation Report: The Nursery Project} 
\author{Aaron Billones, billonea\\Gillian Ford, fordg\\Juan Moncada, moncadaj\\Steven Ramundi, ramundis}
\date{March 8, 2023}
	
\maketitle

\pagenumbering{roman}

\section{Revision History}

\begin{tabularx}{\textwidth}{p{3cm}p{2cm}X}
\toprule {\bf Date} & {\bf Version} & {\bf Notes}\\
\midrule
2023-03-08 & 1.0 & Initial Revision\\

\bottomrule
\end{tabularx}

~\newpage

\section{Symbols, Abbreviations and Acronyms}

\renewcommand{\arraystretch}{1.2}
\begin{tabular}{l l} 
  \toprule		
  \textbf{symbol} & \textbf{description}\\
  \midrule 
  ART & Accessibility Requirements Test\\
CR & Conveyor Functional Requirement\\
CST & Conveyor Subsystem Test\\
EPET & Expected Physical Environment Test\\
LCD & Liquid-Crystal Display\\
LED & Light Emitting Diode\\
LRT & Learning Requirements Test\\
MG & Module Guide\\
MIS & Management Information Systems\\
NFR & Non-Functional Requirement\\
PDST & Pot Dispenser Subsystem Test\\
PDR & Pot Dispensing Functional Requirement\\
PCST & Pot-pulator Complete System Testing\\
PT & Precision Test\\
RT & Reliability Test\\
SCT & Safety Critical Test\\
SRS & Software Requirements Specification\\
SRT & Speed Requirements Test\\
TDST & Tray Dispenser Subsystem Test\\
TDR & Tray Dispensing Functional Requirement\\
VST & Verification Subsystem Test\\
VR & Verification Functional Requirement\\
  \bottomrule
\end{tabular}\\

\newpage

\tableofcontents

\listoftables %if appropriate

\listoffigures %if appropriate

\newpage

\pagenumbering{arabic}

This document ...

\section{Functional Requirements Evaluation}

\subsection{Pot-pulator Complete System Testing}

\begin{table}[h]
  \centering
  \begin{tabularx}{\textwidth}{|X|X|X|X|X|p{1cm}|}
    \hline
    Test Number & Description & Input & Expected Output & Actual Output & Result\\
    \hline
    PCST-01 & Tray Dispenser Operation & Sensor reading of status of tray stack & Normal system operation & Normal system operation & Pass \\
    \hline
    PCST-02 & On Switch for Tray Dispenser Error
  \end{tabularx}
  \caption{PCST Evaluation}
  \label{tab:PCST}
\end{table}

\subsection{Tray Dispenser Subsystem Testing}

\subsection{Pot Dispenser Subsystem Testing}

\subsection{Conveyor Subsystem Testing}

\subsection{Verification Subsystem Testing}

\section{Nonfunctional Requirements Evaluation}

\subsection{Safety Critical Testing}
		
\subsection{Precision Testing}

\subsection{Reliability Testing}

\subsection{Expected Physical Environment Testing}

\subsection{Speed Requirements Testing}

\subsection{Learning Requirements Testing}

\subsection{Accessibility Testing}
	
\section{Comparison to Existing Implementation}	

This section will not be appropriate for every project.

\section{Unit Testing}

\section{Changes Due to Testing}
\subsection{Tray Dispenser Subsystem}

\subsection{Pot Dispenser Subsystem}

\subsection{Conveyor Subsystem}

\subsection{Verification Subsystem}

The bases of the verification mounts were recently expanded away from the conveyer by approximately 5 cm on either side, to address issues with the performance of the ultrasonic sensor along the conveyer belt system. The limitations were identified in the sensor's ability to accurately read values when the trays being measured were too close to the sensor as they moved down the conveyer belt.
\\ The minimum distance that an ultrasonic sensor can measure accurately depends on its design and specifications. In general, most ultrasonic sensors have a minimum range of around 2-3 cm, but this can vary depending on factors such as the sensor's frequency, beam width, and sensitivity. In this application, a 5 cm range worked best. When the first design was implemented, the ultrasonic sensors were mounted directly above the edges of the conveyer belt, resulting in the pots becoming almost directly flush to the sensors when moving down the conveyer belt. In this case, the sensor was not able to effectively detect objects at very close range due to limitations in its design or sensitivity.
\\ To overcome this limitation, the base of the verification mounts was expanded, therefore increasing the distance between the sensor and the trays being measured. This improved the accuracy and reliability of the measurements being collected to ensure that the data generated by the system was of the highest quality possible.


\section{Automated Testing}
		
\section{Trace to Requirements}
		
\section{Trace to Modules}		

\section{Code Coverage Metrics}

\bibliographystyle{plainnat}
\bibliography{../../refs/References}

\newpage{}
\section*{Appendix --- Reflection}

The information in this section will be used to evaluate the team members on the
graduate attribute of Lifelong Learning.  Please answer the following questions:

\begin{enumerate}
  \item 
  \item 
\end{enumerate}

\end{document}