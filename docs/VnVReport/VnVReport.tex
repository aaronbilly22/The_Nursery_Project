\documentclass[12pt, titlepage]{article}

\usepackage{booktabs}
\usepackage{tabularx}
\usepackage{longtable}
\usepackage{hyperref}
\hypersetup{
    colorlinks,
    citecolor=black,
    filecolor=black,
    linkcolor=red,
    urlcolor=blue
}
\usepackage[round]{natbib}

%% Comments

\usepackage{color}

\newif\ifcomments\commentstrue %displays comments
%\newif\ifcomments\commentsfalse %so that comments do not display

\ifcomments
\newcommand{\authornote}[3]{\textcolor{#1}{[#3 ---#2]}}
\newcommand{\todo}[1]{\textcolor{red}{[TODO: #1]}}
\else
\newcommand{\authornote}[3]{}
\newcommand{\todo}[1]{}
\fi

\newcommand{\wss}[1]{\authornote{blue}{SS}{#1}} 
\newcommand{\plt}[1]{\authornote{magenta}{TPLT}{#1}} %For explanation of the template
\newcommand{\an}[1]{\authornote{cyan}{Author}{#1}}

%% Common Parts

\newcommand{\progname}{ProgName} % PUT YOUR PROGRAM NAME HERE
\newcommand{\authname}{Team \#, Team Name
\\ Student 1 name
\\ Student 2 name
\\ Student 3 name
\\ Student 4 name} % AUTHOR NAMES                  

\usepackage{hyperref}
    \hypersetup{colorlinks=true, linkcolor=blue, citecolor=blue, filecolor=blue,
                urlcolor=blue, unicode=false}
    \urlstyle{same}
                                


\begin{document}

\title{Verification and Validation Report: The Nursery Project} 
\author{Aaron Billones, billonea\\Gillian Ford, fordg\\Juan Moncada, moncadaj\\Steven Ramundi, ramundis}
\date{March 8, 2023}
	
\maketitle

\pagenumbering{roman}

\section{Revision History}

\begin{tabularx}{\textwidth}{p{3cm}p{2cm}X}
\toprule {\bf Date} & {\bf Version} & {\bf Notes}\\
\midrule
2023-03-08 & 1.0 & Initial Revision\\

\bottomrule
\end{tabularx}

~\newpage

\section{Symbols, Abbreviations and Acronyms}

\renewcommand{\arraystretch}{1.2}
\begin{tabular}{l l} 
  \toprule		
  \textbf{symbol} & \textbf{description}\\
  \midrule 
  ART & Accessibility Requirements Test\\
CR & Conveyor Functional Requirement\\
CST & Conveyor Subsystem Test\\
EPET & Expected Physical Environment Test\\
LCD & Liquid-Crystal Display\\
LED & Light Emitting Diode\\
LRT & Learning Requirements Test\\
MG & Module Guide\\
MIS & Management Information Systems\\
NFR & Non-Functional Requirement\\
PDST & Pot Dispenser Subsystem Test\\
PDR & Pot Dispensing Functional Requirement\\
PCST & Pot-pulator Complete System Testing\\
PT & Precision Test\\
RT & Reliability Test\\
SCT & Safety Critical Test\\
SRS & Software Requirements Specification\\
SRT & Speed Requirements Test\\
TDST & Tray Dispenser Subsystem Test\\
TDR & Tray Dispensing Functional Requirement\\
VST & Verification Subsystem Test\\
VR & Verification Functional Requirement\\
  \bottomrule
\end{tabular}\\

\newpage

\tableofcontents

\listoftables %if appropriate

\listoffigures %if appropriate

\newpage

\pagenumbering{arabic}

\section{Functional Requirements Evaluation}

\subsection{Pot-pulator Complete System Testing}

\begin{table}[h]
  \centering
  \begin{tabularx}{\textwidth}{|X|X|X|X|X|p{1cm}|}
    \hline
    Test Number & Description & Input & Expected Output & Actual Output & Result\\
    \hline
    PCST-01 & Tray Dispenser Operation & Sensor reading of status of tray stack & Normal system operation & Normal system operation & Pass \\
    \hline
    PCST-02 & On Switch for Tray Dispenser Error & Pot-pulator switch set to on & Normal system operation & Normal system operation & Pass\\
    \hline
    PCST-03 & Pot Dispenser Operation & Sensor reading of status of pot stack & Normal system operation & Normal system operation & Pass\\
    \hline
    PCST-04 & On Switch for Pot Dispenser Error & Pot-pulator switch set to on & Normal system operation & Normal system operation & Pass\\
    \hline
    PCST-05 & Conveyor Operation & Tray placed on conveyor &  Normal system operation & Normal system operation & Pass\\
    \hline
    PCST-06 & 
  \end{tabularx}
  \caption{PCST Evaluation}
  \label{tab:PCST}
\end{table}

\subsection{Tray Dispenser Subsystem Testing}

\subsection{Pot Dispenser Subsystem Testing}

\subsection{Conveyor Subsystem Testing}

\subsection{Verification Subsystem Testing}

\section{Nonfunctional Requirements Evaluation}

\subsection{Safety Critical Testing}
		
\subsection{Precision Testing}

\subsection{Reliability Testing}

\subsection{Expected Physical Environment Testing}

\subsection{Speed Requirements Testing}

\subsection{Learning Requirements Testing}

\subsection{Accessibility Testing}
	
\section{Comparison to Existing Implementation}	

This section will not be appropriate for every project.

\section{Unit Testing}

\section{Changes Due to Testing}

The changes due to testing are summarized below:
\\
1. The bases of the verification mounts were recently expanded away from the
conveyer by approximately 5 cm on either side, to address issues with the pots being too close to the sensor when travelling down the conveyer, resulting in inaccurate readings.

\section{Automated Testing}
		
\section{Trace to Requirements}

The following table outlines all of the system tests and how they relate to the
relevent requirements. The requirements can be referenced in the SRS document.\\

\begin{longtable}{ |p{4cm}|p{8cm}|  }
  \caption{Corresponding Test IDs and Requirements}
  \label{tab:Table1}\\
  
  \hline
  \textbf{Test ID} & \textbf{Supporting Requirements}\\
  \hline
  TDST-01 &  TDR3, TDR5\\
  \hline
  TDST-02 &  TDR4, TDR5 \\
  \hline
  TDST-03 &  TDR2 \\
  \hline
  TDST-04 &  TDR2 \\
  \hline
  PDST-01 &  PDR2 \\
  \hline
  PDST-02 &  PDR2 \\
  \hline
  PDST-03 &  PDR3 \\
  \hline
  PDST-04 &  PDR4 \\
  \hline
  PDST-05 &  PDR5, PDR6 \\
  \hline
  PCST-01 &  TDR1 \\
  \hline
  PCST-02 &  TDR5, TDR6 \\
  \hline
  PCST-03 &  TDR7 \\
  \hline
  PCST-04 &  PDR1 \\
  \hline
  PCST-05 &  PDR6, PDR7 \\
  \hline
  PCST-06 &  PDR8 \\
  \hline
  PCST-07 &  CR1 \\
  \hline
  PCST-08 &  CR5 \\
  \hline
  PCST-09 &  CR6 \\
  \hline
  VST-01 &  VR1 \\
  \hline
  VST-02&  VR2 \\
  \hline
  SCT-01 &  NFR12 \\
  \hline
  SCT-02 &  NFR12 \\
  \hline
  SCT-03 &  NFR12 \\
  \hline
  SCT-04 &  NFR12 \\
  \hline
  PT-01 & NFR13 \\
  \hline
  PT-02 & NFR14 \\
  \hline
  RT-01 & NFR17 \\
  \hline
  EPET-01 & NFR20 \\
  \hline
  LRT-01&  NFR6 \\
  \hline
  ART-01 & NFR7 \\
  \hline
  SRT-01 & NFR8 \\
  \hline
  SRT-02 & NFR9 \\
  \hline
  SRT-03 & NFR10 \\
  \hline
\end{longtable}

\newpage

		
\section{Trace to Modules}	

The following table outlines all of the system tests and how they relate to the
relevent modules. The modules can be referenced in the MG document.\\

\begin{longtable}{ |p{4cm}|p{8cm}|  }
  \caption{Corresponding Test IDs and Requirements}
  \label{tab:Table1}\\
  
  \hline
  \textbf{Test ID} & \textbf{Supporting Requirements}\\
  \hline
  TDST-01 &  \\
  \hline
  TDST-02 &   \\
  \hline
  TDST-03 &   \\
  \hline
  TDST-04 &   \\
  \hline
  PDST-01 &   \\
  \hline
  PDST-02 &   \\
  \hline
  PDST-03 &   \\
  \hline
  PDST-04 &   \\
  \hline
  PDST-05 &  \\
  \hline
  PCST-01 &   \\
  \hline
  PCST-02 &  \\
  \hline
  PCST-03 &   \\
  \hline
  PCST-04 &   \\
  \hline
  PCST-05 &  \\
  \hline
  PCST-06 &   \\
  \hline
  PCST-07 &   \\
  \hline
  PCST-08 &   \\
  \hline
  PCST-09 &   \\
  \hline
  VST-01 &   \\
  \hline
  VST-02&   \\
  \hline
  SCT-01 &   \\
  \hline
  SCT-02 &   \\
  \hline
  SCT-03 &   \\
  \hline
  SCT-04 &   \\
  \hline
  PT-01 &  \\
  \hline
  PT-02 &  \\
  \hline
  RT-01 &  \\
  \hline
  EPET-01 &  \\
  \hline
  LRT-01&   \\
  \hline
  ART-01 &  \\
  \hline
  SRT-01 &  \\
  \hline
  SRT-02 &  \\
  \hline
  SRT-03 &  \\
  \hline
\end{longtable}

\newpage

\section{Code Coverage Metrics}

\bibliographystyle{plainnat}
\bibliography{../../refs/References}

\newpage{}
\section*{Appendix --- Reflection}

The information in this section will be used to evaluate the team members on the
graduate attribute of Lifelong Learning.  Please answer the following questions:

\begin{enumerate}
  \item 
  \item 
\end{enumerate}

\end{document}