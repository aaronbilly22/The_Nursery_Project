\documentclass{article}

\usepackage{booktabs}
\usepackage{tabularx}
\usepackage{hyperref}
\usepackage[english]{babel}
\usepackage{enumitem}
\usepackage{pdflscape}
\usepackage{array}
\usepackage{longtable}
\hypersetup{
    colorlinks=true,       % false: boxed links; true: colored links
    linkcolor=red,          % color of internal links (change box color with linkbordercolor)
    citecolor=green,        % color of links to bibliography
    filecolor=magenta,      % color of file links
    urlcolor=cyan           % color of external links
}

\newcounter{srreqnum} %Likely change number
\newcommand{\lthesrreqnum}{LC\thesrreqnum}
\newcommand{\srref}[1]{SR\ref{#1}}


\title{Hazard Analysis The Nursery Poject}

\author{Aaron Billones, billonea\\Gillian Ford, fordg\\Juan Moncada, moncadaj\\Steven Ramundi, ramundis}

\date{October 19, 2022}

%% Comments

\usepackage{color}

\newif\ifcomments\commentstrue %displays comments
%\newif\ifcomments\commentsfalse %so that comments do not display

\ifcomments
\newcommand{\authornote}[3]{\textcolor{#1}{[#3 ---#2]}}
\newcommand{\todo}[1]{\textcolor{red}{[TODO: #1]}}
\else
\newcommand{\authornote}[3]{}
\newcommand{\todo}[1]{}
\fi

\newcommand{\wss}[1]{\authornote{blue}{SS}{#1}} 
\newcommand{\plt}[1]{\authornote{magenta}{TPLT}{#1}} %For explanation of the template
\newcommand{\an}[1]{\authornote{cyan}{Author}{#1}}

%% Common Parts

\newcommand{\progname}{ProgName} % PUT YOUR PROGRAM NAME HERE
\newcommand{\authname}{Team \#, Team Name
\\ Student 1 name
\\ Student 2 name
\\ Student 3 name
\\ Student 4 name} % AUTHOR NAMES                  

\usepackage{hyperref}
    \hypersetup{colorlinks=true, linkcolor=blue, citecolor=blue, filecolor=blue,
                urlcolor=blue, unicode=false}
    \urlstyle{same}
                                


\begin{document}

\maketitle
\thispagestyle{empty}



\pagenumbering{roman}

\begin{tabularx}{\textwidth}{p{3cm}p{4cm}X}
    \toprule {\bf Date} & {\bf Version} & {\bf Notes}\\
    \midrule
    2022-10-19 & Juan Moncada,& Initial release\\&Aaron Billones,\\&Steven Ramundi,\\&Gillian Ford \\
    
    \bottomrule
\end{tabularx}

~\newpage

\tableofcontents

~\newpage

\pagenumbering{arabic}

\section{Introduction}

\noindent This document is the hazard analysis of the Pot-Pulator. This machine will be built for Sheridan Nurseries, filling trays with pots, preparing them to be populated with soil and seeds, to reduce the manual labour of the workers at the farm. The hazard analysis will consider each part of the Pot-Pulator, including the respective tray and pot droppers, the sensor verification section, and the conveyer belt. 
\\

\noindent The definition of a hazard used in this document is any issue or property in the Pot-Pulator that will cause a risk to the ideal result of the system. In this system, most hazards will be concerning safety of the workers collecting the trays at the end of the conveyer, and the effectiveness of the verification system. 
\\

\noindent This document will include the Scope and Purpose of Hazard Analysis, System Boundaries and Components, Failure Mode and Effect Analysis, and Safety \& Security Requirements.  

\section{Scope and Purpose of Hazard Analysis}

\noindent The scope of this document is to identify the components of the Pot-Pulator that could have harmful consequences to the users or the results and reduce each risk to a level where the overall system will be safe and acceptable. 
\\

\noindent Hazards will be analyzed based on research of similar systems, and any specific hazards occurring throughout the development process of the machine. 

\section{System Boundaries and Components}

The hazard analysis will be conducted based on the Pot-Pulator, which is restricted to the components below. 
\\

\begin{enumerate}
    \item Conveyer belt
    \item Tray allocation
    \item Pot dropping
    \item Verification system
\end{enumerate}

\noindent The hazard analysis will be based on these critical elements of the system. The system boundary includes a conveyer belt moving the system along, tray allocation to place the trays on the conveyer, a pot dropper to place the pots into the trays, and a verification system to confirm the correct placement of the components. 
\clearpage
\noindent The conveyer belt will be controlled by a sensor based on the other components. At the end of the process, the conveyer belt will bring the completed trays to the end of the line, to be collected by a worker. This will all be accounted for in the hazard analysis. 

\section{Critical Assumptions}

There are no critical assumptions for this system.




\newpage
\begin{landscape}
\section{Failure Mode and Effect Analysis}

\begin{center}
    \begin{longtable}{|l|  p{3cm}  p{4cm}  p{4cm}  p{4cm}  p{1cm}  p{1cm}|}
        \hline
        Component&
        Failure Mode&
        Effect of Failure&
        Cause of failure &
        Recomended action &
        SR&
        Ref \\
        \hline
        \begin{centering}
        Tray Dispensing
        \end{centering} &
        Tray is not dispensed & 
        Machine is unable to continue operation, tray may be damaged & 
              \begin{enumerate}[label=(\alph*)]
                  \item Tray stack software/hardware failure
                  \item Tray dispenser software/hardware failure 
                  \item Parts failure 
              \end{enumerate} &
              \begin{enumerate}[label=(\alph*)]
                  \item  Sensor will recognize if tray has not been dispensed, error message will be displayed and operator will be notified.
                  \item  Refer to H1-1a
                  \item Refer to H1-1a
              \end{enumerate}&
          &
          H1-1\\
 
         &
        Trays placed incorrectly on conveyor&
        Tray is unable to move forward on conveyor, pot dispenser is unable to place pots correctly. May damage pots/trays&
        Tray dispenser software/hardware failure&
        Guiding rods will be placed on the conveyor to centre trays into correct position. If trays are unable to move forward, error message will be displayed and operator will be notified.&
        &
        H1-2\\

        &
        Trays dispensed are stacked, 2+ trays dispensed at once&
        Tray storage becomes out of sync with pot storage, may damage pots/trays&
            \begin{enumerate}
             \item Parts failure 
             \item Trays loaded incorrectly
            \end{enumerate} &
            \begin{enumerate}
              \item Sensor will recognize if multiple trays have been dispensed, error message will be displayed and operator will be notified
              \item Operator will be trained to properly load trays into machine"	
            \end{enumerate}&
        &
        H1-3\\
        &
        Tray dispenser damages tray&	
        Tray is unable to hold pots and be sent for distribution (will depend on severity of damage to tray)&
        Quality issue in trays&
        Operator will be trained to perform 60 second visual check of Pot-pulator, trays, and pots before each refill to note and eliminate trays with any noticeable defects&		
        &
        H1-4\\
        \hline
        Pot Dispenser&
        Pots are not dispensed&
        Tray will be dispensed empty&
        Software/hardware failure&
        Sensor post pot dispensing will sense that the tray has not been populated, error message will be displayed and operator will be notified&		
        &
        H2-1\\
        \hline
        Conveyor&
        Conveyor does not move&
        Tray is unable to move to soil filling machine, trays may be damaged&
        \begin{enumerate}[label=(\alph*)]
            \item Software/hardware failure
            \item Parts failure
        \end{enumerate}&
        \begin{enumerate}[label=(\alph*)]
            \item Sensor will recognize if conveyor is not moving for extended period of time, error message will be displayed and operator will be notified
            \item Operator will be trained to perform 60 second visual check of Pot-pulator before pots and trays refill to note any damage to conveyor or belt wear
        \end{enumerate}&
        &
        H3-1\\
        &
        Conveyor does not stop when it is meant to&
        Pots are unable to be placed properly in trays, pots and/or trays may be damaged&
        Software/hardware failure&
        If tray is in view of pot dispenser, refer to H2-2a. If tray is not in view of pot dispenser, refer to H2-3&
        &
        H3-2\\
        &
        Trays slide on conveyor when conveyor is accelerating/deccelerating&
        Pots are unable to be placed properly in trays, trays may be damaged&
        \begin{enumerate}[label=(\alph*)]
            \item Software/hardware failure
            \item Belt failure
        \end{enumerate}&
        \begin{enumerate}[label=(\alph*)]
            \item Sensor will recognize if conveyor is accelerating/deccelerating at a magnitude greater than specified, error message will be displayed and operator will be notified
            \item Refer to H3-1b
        \end{enumerate}&
        &
        H3-3\\

        &
        Conveyor is unable to reach desired speed&
        Machine is unable to meet production standards&
        \begin{enumerate}[label=(\alph*)]
            \item Software/hardware failure
            \item Parts failure
        \end{enumerate}&
        \begin{enumerate}[label=(\alph*)]
            \item  Sensor will recognize if conveyor is not reaching desired speed for extended period of time, error message will be displayed and operator will be notified
            \item Refer to H3-1b
        \end{enumerate}&
        &
        H3-4\\



    \cline{2-7}
    \hline

    % Conveyor & Failure Mode & Effect of Failure & Cause of failure & Recomended action \\
    % \hline
    % Pot Dispensing & Failure Mode & Effect of Failure & Cause of failure & Recomended action  \\
    % \hline
    % Verification & Failure Mode & Effect of Failure & Cause of failure & Recomended action  \\
    % \hline
    \end{longtable}
\end{center}
\end{landscape}


\newpage

\section{Safety and Security Requirements}

\subsection{Existing Requirements}
The following are listed as non-functional requirements that can be found in 
the Software Requirements Specification (SRS) document (reference in square brackets).
These requirements are related to the safety and security of the system which are necessary during operation.
\\
\begin{itemize}
    \item[SR\refstepcounter{srreqnum}\thesrreqnum \label{SR1}:]
    All electrical equipment and electronics must be well covered and protected. The user must not have access to equipment. [NFR1]
    \item[SR\refstepcounter{srreqnum}\thesrreqnum \label{SR2}:]
    All wiring must be tucked away and not accessible to avoid potential electrical failure. [NFR2]
    \item[SR\refstepcounter{srreqnum}\thesrreqnum \label{SR3}:]
    All moving parts must be covered and protected. Moving parts should be covered to protect both the mechanism and the safety of the operator. [NFR3]
    \item[SR\refstepcounter{srreqnum}\thesrreqnum \label{SR4}:]
    System must have both audible and visible signal outputs for each system status. [NFR7]
    \item[SR\refstepcounter{srreqnum}\thesrreqnum \label{SR5}:]
    Conveyor system must not accelerate in a manner that would shift the position of 
    the tray on the conveyor belt. A shift in the position of the tray could result in a misalignment and potential error. [NFR8]
    \item[SR\refstepcounter{srreqnum}\thesrreqnum \label{SR6}:]
    System must have emergency cut off. In case of any emergency, this will trip off all power to the Pot-Pulator. [NFR11]
    \item[SR\refstepcounter{srreqnum}\thesrreqnum \label{SR7}:]
    System must be able to locate and identify failures within each independent subsystem. [NFR12]
    \item[SR\refstepcounter{srreqnum}\thesrreqnum \label{SR8}:]
    SR8: System must follow electronic component safety requirements. [NFR25]
    
  \end{itemize}

\subsection{New Requirements}
The following are listed as non-functional requirements related to 
safety and security of the system that were formulated upon performing 
hazard analysis. Reference to the new requirements in the SRS are shown in square brackets.\\

\begin{itemize}
    \item[SR\refstepcounter{srreqnum}\thesrreqnum \label{SR9}:]
    System must be able to withstand a sufficient physical force to keep from falling over. [NFR26]
    \item[SR\refstepcounter{srreqnum}\thesrreqnum \label{SR10}:]
    System must obtain a user login system or physical start key. [NFR27]
    \item[SR\refstepcounter{srreqnum}\thesrreqnum \label{SR11}:]
    Clear physical emergency instructions and labels located on the device. [NFR28]
    \item[SR\refstepcounter{srreqnum}\thesrreqnum \label{SR12}:]
    System must have surge protection. [NFR29]
\end{itemize}

\section{Roadmap}

There were some new non-functional requirements (safety and security) discovered, 
after performing hazard analysis on the system. The safety and security requirements 
that will be implemented in the future beyond the timeline of the capstone are SR7, 
SR10, and SR12. These requirements will be apart of a future implementation due to 
time-related and financial reasons. The remaining non-functional requirements related 
to safety and security will be implemented within the timeline the capstone project.

\end{document}