\documentclass{article}

\usepackage{booktabs}
\usepackage{tabularx}
\usepackage{hyperref}
\usepackage[english]{babel}
\usepackage{enumitem}
\usepackage{pdflscape}
\usepackage{array}
\usepackage{longtable}
\hypersetup{
    colorlinks=true,       % false: boxed links; true: colored links
    linkcolor=red,          % color of internal links (change box color with linkbordercolor)
    citecolor=green,        % color of links to bibliography
    filecolor=magenta,      % color of file links
    urlcolor=cyan           % color of external links
}

\title{Hazard Analysis The Nursery Poject}

\author{\authname}

\date{}

%% Comments

\usepackage{color}

\newif\ifcomments\commentstrue %displays comments
%\newif\ifcomments\commentsfalse %so that comments do not display

\ifcomments
\newcommand{\authornote}[3]{\textcolor{#1}{[#3 ---#2]}}
\newcommand{\todo}[1]{\textcolor{red}{[TODO: #1]}}
\else
\newcommand{\authornote}[3]{}
\newcommand{\todo}[1]{}
\fi

\newcommand{\wss}[1]{\authornote{blue}{SS}{#1}} 
\newcommand{\plt}[1]{\authornote{magenta}{TPLT}{#1}} %For explanation of the template
\newcommand{\an}[1]{\authornote{cyan}{Author}{#1}}

%% Common Parts

\newcommand{\progname}{ProgName} % PUT YOUR PROGRAM NAME HERE
\newcommand{\authname}{Team \#, Team Name
\\ Student 1 name
\\ Student 2 name
\\ Student 3 name
\\ Student 4 name} % AUTHOR NAMES                  

\usepackage{hyperref}
    \hypersetup{colorlinks=true, linkcolor=blue, citecolor=blue, filecolor=blue,
                urlcolor=blue, unicode=false}
    \urlstyle{same}
                                


\begin{document}

\maketitle
\thispagestyle{empty}

~\newpage

\pagenumbering{roman}

\begin{table}[hp]
\caption{Revision History} \label{TblRevisionHistory}
\begin{tabularx}{\textwidth}{llX}
\toprule
\textbf{Date} & \textbf{Developer(s)} & \textbf{Change}\\
\midrule
Date1 & Name(s) & Description of changes\\
Date2 & Name(s) & Description of changes\\
... & ... & ...\\
\bottomrule
\end{tabularx}
\end{table}

~\newpage

\tableofcontents

~\newpage

\pagenumbering{arabic}

\section{Introduction}

\noindent This document is the hazard analysis of the Pot-Pulator. This machine will be built for Sheridan Nurseries, filling trays with pots, preparing them to be populated with soil and seeds, to reduce the manual labour of the workers at the farm. The hazard analysis will consider each part of the Pot-Pulator, including the respective tray and pot droppers, the sensor verification section, and the conveyer belt. 
\\

\noindent The definition of a hazard used in this document is any issue or property in the Pot-Pulator that will cause a risk to the ideal result of the system. In this system, most hazards will be concerning safety of the workers collecting the trays at the end of the conveyer, and the effectiveness of the verification system. 
\\

\noindent This document will include the Scope and Purpose of Hazard Analysis, System Boundaries and Components, Failure Mode and Effect Analysis, and Safety & Security Requirements.  

\section{Scope and Purpose of Hazard Analysis}

\noindent The scope of this document is to identify the components of the Pot-Pulator that could have harmful consequences to the users or the results and reduce each risk to a level where the overall system will be safe and acceptable. 
\\

\noindent Hazards will be analyzed based on research of similar systems, and any specific hazards occurring throughout the development process of the machine. 

\section{System Boundaries and Components}

The hazard analysis will be conducted based on the Pot-Pulator, which is restricted to the components below. 
\\

\indent 1. Conveyer belt
\\

2.	Tray allocation
\\

3.	Pot dropping
\\

4.	Verification system
\\

\noindent The hazard analysis will be based on these critical elements of the system. The system boundary includes a conveyer belt moving the system along, tray allocation to place the trays on the conveyer, a pot dropper to place the pots into the trays, and a verification system to confirm the correct placement of the components. 
\clearpage
\noindent The conveyer belt will be controlled by a sensor based on the other components. At the end of the process, the conveyer belt will bring the completed trays to the end of the line, to be collected by a worker. This will all be accounted for in the hazard analysis. 

\section{Critical Assumptions}

There are no critical assumptions for this system.




\newpage
\begin{landscape}
\section{Failure Mode and Effect Analysis}

\begin{center}
    \begin{longtable}{|l|  p{3cm}  p{4cm}  p{4cm}  p{4cm}  p{1cm}  p{1cm}|}
        \hline
        Component&
        Failure Mode&
        Effect of Failure&
        Cause of failure &
        Recomended action &
        SR&
        Ref \\
        \hline
        \begin{centering}
        Tray Dispensing
        \end{centering} &
        Tray is not dispensed & 
        Machine is unable to continue operation, tray may be damaged & 
              \begin{enumerate}[label=(\alph*)]
                  \item Tray stack software/hardware failure
                  \item Tray dispenser software/hardware failure 
                  \item Parts failure 
              \end{enumerate} &
              \begin{enumerate}[label=(\alph*)]
                  \item  Sensor will recognize if tray has not been dispensed, error message will be displayed and operator will be notified.
                  \item  Refer to H1-1a
                  \item Refer to H1-1a
              \end{enumerate}&
          &
          H1-1\\
 
         &
        Trays placed incorrectly on conveyor&
        Tray is unable to move forward on conveyor, pot dispenser is unable to place pots correctly. May damage pots/trays&
        Tray dispenser software/hardware failure&
        Guiding rods will be placed on the conveyor to centre trays into correct position. If trays are unable to move forward, error message will be displayed and operator will be notified.&
        &
        H1-2\\

        &
        Trays dispensed are stacked, 2+ trays dispensed at once&
        Tray storage becomes out of sync with pot storage, may damage pots/trays&
            \begin{enumerate}
             \item Parts failure 
             \item Trays loaded incorrectly
            \end{enumerate} &
            \begin{enumerate}
              \item Sensor will recognize if multiple trays have been dispensed, error message will be displayed and operator will be notified
              \item Operator will be trained to properly load trays into machine"	
            \end{enumerate}&
        &
        H1-3\\
        &
        Tray dispenser damages tray&	
        Tray is unable to hold pots and be sent for distribution (will depend on severity of damage to tray)&
        Quality issue in trays&
        Operator will be trained to perform 60 second visual check of Pot-pulator, trays, and pots before each refill to note and eliminate trays with any noticeable defects&		
        &
        H1-4\\
        \hline
        Pot Dispenser&
        Pots are not dispensed&
        Tray will be dispensed empty&
        Software/hardware failure&
        Sensor post pot dispensing will sense that the tray has not been populated, error message will be displayed and operator will be notified&		
        &
        H2-1\\



    \cline{2-7}
    \hline

    % Conveyor & Failure Mode & Effect of Failure & Cause of failure & Recomended action \\
    % \hline
    % Pot Dispensing & Failure Mode & Effect of Failure & Cause of failure & Recomended action  \\
    % \hline
    % Verification & Failure Mode & Effect of Failure & Cause of failure & Recomended action  \\
    % \hline
    \end{longtable}
\end{center}
\end{landscape}


\newpage

\section{Safety and Security Requirements}

\wss{Newly discovered requirements.  These should also be added to the SRS.  (A
rationale design process how and why to fake it.)}

\section{Roadmap}

\wss{Which safety requirements will be implemented as part of the capstone timeline?
Which requirements will be implemented in the future?}

\end{document}