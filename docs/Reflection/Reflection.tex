\documentclass{article}

\usepackage{tabularx}
\usepackage{booktabs}

\title{Reflection Report on Pot-pulator}

\author{Aaron Billones, billonea\\Gillian Ford, fordg\\Juan Moncada, moncadaj\\Steven Ramundi, ramundis}

\date{April 4, 2023}

%% Comments

\usepackage{color}

\newif\ifcomments\commentstrue %displays comments
%\newif\ifcomments\commentsfalse %so that comments do not display

\ifcomments
\newcommand{\authornote}[3]{\textcolor{#1}{[#3 ---#2]}}
\newcommand{\todo}[1]{\textcolor{red}{[TODO: #1]}}
\else
\newcommand{\authornote}[3]{}
\newcommand{\todo}[1]{}
\fi

\newcommand{\wss}[1]{\authornote{blue}{SS}{#1}} 
\newcommand{\plt}[1]{\authornote{magenta}{TPLT}{#1}} %For explanation of the template
\newcommand{\an}[1]{\authornote{cyan}{Author}{#1}}

%% Common Parts

\newcommand{\progname}{ProgName} % PUT YOUR PROGRAM NAME HERE
\newcommand{\authname}{Team \#, Team Name
\\ Student 1 name
\\ Student 2 name
\\ Student 3 name
\\ Student 4 name} % AUTHOR NAMES                  

\usepackage{hyperref}
    \hypersetup{colorlinks=true, linkcolor=blue, citecolor=blue, filecolor=blue,
                urlcolor=blue, unicode=false}
    \urlstyle{same}
                                


\begin{document}

\maketitle

% \plt{Reflection is an important component of getting the full benefits from a
% learning experience.  Besides the intrinsic benefits of reflection, this
% document will be used to help the TAs grade how well your team responded to
% feedback.  In addition, several CEAB (Canadian Engineering Accreditation Board)
% Learning Outcomes (LOs) will be assessed based on your reflections.}

\section{Changes in Response to Feedback}

% \plt{Summarize the changes made over the course of the project in response to
% feedback from TAs, the instructor, teammates, other teams, the project
% supervisor (if present), and from user testers.}

% \plt{For those teams with an external supervisor, please highlight how the feedback 
% from the supervisor shaped your project.  In particular, you should highlight the 
% supervisor's response to your Rev 0 demonstration to them.}

\subsection{SRS and Hazard Analysis}

Changes were made to SRS based on feedback from teammates, the instructor, and TAs. 
Requirements were adjusted as the design and scope of the project changed.

\subsection{Design and Design Documentation}

Design underwent many major changes throughout the process based on feedback. Between Proof of Concept 
and Rev0 Demo, feedback from teammates influenced a major change in the final design of the tray dropper 
subsystem. After Rev0, feeback from the instructor and TA influenced the revised design of the pot dropper 
subsystem, increasing the accuracy of the machine overall.

\subsection{VnV Plan and Report}

The VnV Plan and Report changed based on feedback from peers. Testing plans were updated to better ensure all 
components of the machine were being tested thoroughly, and all edge cases were considered when conducting testing.
\section{Design Iteration (LO11)}

\plt{Explain how you arrived at your final design and implementation.  How did
the design evolve from the first version to the final version?} 

\section{Design Decisions (LO12)}

\plt{Reflect and justify your design decisions.  How did limitations,
 assumptions, and constraints influence your decisions?}

\section{Economic Considerations (LO23)}

% \plt{Is there a market for your product? What would be involved in marketing your 
% product? What is your estimate of the cost to produce a version that you could 
% sell?  What would you charge for your product?  How many units would you have to 
% sell to make money? If your product isn't something that would be sold, like an 
% open source project, how would you go about attracting users?  How many potential 
% users currently exist?}

We believe there is a market for our product. All Sheridan Nurseries farms are ran in the same way 
as the farm which we used to influence our design. The cost of the manual labour does not justify the 
purchase of the machines that are currently available on the market, but our low cost solution is perfect 
for Sheridan Nurseries' current situation. It is also a great solution for any industrial potting farm which 
owns a soil and seed filling machine, but does not own a pot and tray sorting machine. Marketing the product 
would involve giving live demonstrations to decision makers at the farms or distributing videos showcasing the functionality 
of the machine and highlighting the value it will provide. We estimate it would cost approximately \$700 to manufacture 
if our current sourcing methods are used, but this price can be drastically reduced if it were to be manufactured in large 
quantities. We would charge a price which would translate to a 60\% markup. At a cost of \$700, this would mean setting the price 
at approximately \$1150. We would be making about \$450 per machine, and our break even point would be dependent on the amount of 
capital we would be required to invest to establish a manufacturing process, and the amount of fixed costs that would incur on other 
activities such as advertising.

\section{Reflection on Project Management (LO24)}

% \plt{This question focuses on processes and tools used for project management.}


\subsection{How Does Your Project Management Compare to Your Development Plan}

% \plt{Did you follow your Development plan, with respect to the team meeting plan, 
% team communication plan, team member roles and workflow plan.  Did you use the 
% technology you planned on using?}

The project resembled the structure outlined in the development plan well. meetings were set, updates were given regularly, and each member was held accountable for their contributions to the project, this has a direct correlation with setting reasonable goals in the development plan and not shooting for the moon with meetings and objectives.

\subsection{What Went Well?}

% \plt{What went well for your project management in terms of processes and 
% technology?}

Overall, the design and integration of the capstone went well, dividing the project into subsections when surprisingly well and each member being responsible for their subsection with standardized communication made integration run smoothly.

\subsection{What Went Wrong?}

% \plt{What went wrong in terms of processes and technology?}

A consideration that was overlooked at the beginning of the project was the amount of impact that precision would have on the project, while sections and parts were designed to be as accurate as possible, the designs relied on perfect alignment and sometimes ideal situations, these where not accounted for in the design.

\subsection{What Would you Do Differently Next Time?}

% \plt{What will you do differently for your next project?}

If we had the opportunity to go back and change our project we would focus much more on developing a more redundant machine that relied less on accuracy and precision. We would also set smaller goals and set aside more time to integrate separate subsystems and tuning.

\end{document}