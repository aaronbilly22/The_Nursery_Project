\documentclass{article}

\usepackage{tabularx}
\usepackage{booktabs}

\title{Reflection Report on Pot-pulator}

\author{Team \#24, The Nursery Project\\Aaron Billones, billonea\\Gillian Ford, fordg\\Juan Moncada, moncadaj\\Steven Ramundi, ramundis}
\date{April 5, 2023}

\input{../Comments}
\input{../Common}

\begin{document}

\maketitle

\plt{Reflection is an important component of getting the full benefits from a
learning experience.  Besides the intrinsic benefits of reflection, this
document will be used to help the TAs grade how well your team responded to
feedback.  In addition, several CEAB (Canadian Engineering Accreditation Board)
Learning Outcomes (LOs) will be assessed based on your reflections.}

\section{Changes in Response to Feedback}

\plt{Summarize the changes made over the course of the project in response to
feedback from TAs, the instructor, teammates, other teams, the project
supervisor (if present), and from user testers.}

\plt{For those teams with an external supervisor, please highlight how the feedback 
from the supervisor shaped your project.  In particular, you should highlight the 
supervisor's response to your Rev 0 demonstration to them.}

\subsection{SRS and Hazard Analysis}

\subsection{Design and Design Documentation}

\subsection{VnV Plan and Report}

\section{Design Iteration (LO11)}

The design process of the Pot-pulator went through various iterations.
Our initial design for inserting the pots into the trays consisted of 
actuators that would place the pots inside the slots. While this solution 
is more precise and robust than our final solution, it requires very expensive
materials in order to have it fully operational. After confirming our design for dispensing the pots,
there were iterations with our 3-D printed parts that improved with each iteration.
At first the thread in the disc was too wide where the device would drop multiple pots at one time.
Therefore, we went through several design iterations in order to achieve the correct part design.\\\\

\noindent Similar to the pot dispenser, the tray dispenser design had been through multiple iterations.
Initially, we were going to use and gantry system to move the trays from a stack to the conveyor belt.
We soon came to realize that this solution would also be very expensive and lacking the time/materials in 
order to obtain a working final solution. After we iterated through our gear design which drops the trays
from a stack above the conveyor one at a time. This also took some iterating as we needed to ensure that only one tray 
would be dropped when we want it to.

\section{Design Decisions (LO12)}

With our project being more of a physical solution rather than a software one, 
there were many assumptions made and a lot more constraints pressed upon us.
For example, the 750 dollar budget limited us to changing some of our designs because some initial ones were too expensive.
Some assumptions made were that the area would be clean and weather would not affect our device.
Weather proofing our device would cost more money and materials that we did not have access to.
We also should have made more timely design decisions as there were many instances where we were falling behind.
More timely design goals should have been set which would help our overall design process and success in the project.

\section{Economic Considerations (LO23)}

\plt{Is there a market for your product? What would be involved in marketing your 
product? What is your estimate of the cost to produce a version that you could 
sell?  What would you charge for your product?  How many units would you have to 
sell to make money? If your product isn't something that would be sold, like an 
open source project, how would you go about attracting users?  How many potential 
users currently exist?}

\section{Reflection on Project Management (LO24)}

\plt{This question focuses on processes and tools used for project management.}

\subsection{How Does Your Project Management Compare to Your Development Plan}

\plt{Did you follow your Development plan, with respect to the team meeting plan, 
team communication plan, team member roles and workflow plan.  Did you use the 
technology you planned on using?}

\subsection{What Went Well?}

\plt{What went well for your project management in terms of processes and 
technology?}

\subsection{What Went Wrong?}

\plt{What went wrong in terms of processes and technology?}

\subsection{What Would you Do Differently Next Time?}

\plt{What will you do differently for your next project?}

\end{document}