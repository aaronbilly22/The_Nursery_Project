\documentclass[12pt, titlepage]{article}

\usepackage{booktabs}
\usepackage{tabularx}
\usepackage{hyperref}
\usepackage{longtable}
\hypersetup{
    colorlinks,
    citecolor=blue,
    filecolor=black,
    linkcolor=red,
    urlcolor=blue
}
\usepackage[round]{natbib}

%% Comments

\usepackage{color}

\newif\ifcomments\commentstrue %displays comments
%\newif\ifcomments\commentsfalse %so that comments do not display

\ifcomments
\newcommand{\authornote}[3]{\textcolor{#1}{[#3 ---#2]}}
\newcommand{\todo}[1]{\textcolor{red}{[TODO: #1]}}
\else
\newcommand{\authornote}[3]{}
\newcommand{\todo}[1]{}
\fi

\newcommand{\wss}[1]{\authornote{blue}{SS}{#1}} 
\newcommand{\plt}[1]{\authornote{magenta}{TPLT}{#1}} %For explanation of the template
\newcommand{\an}[1]{\authornote{cyan}{Author}{#1}}

%% Common Parts

\newcommand{\progname}{ProgName} % PUT YOUR PROGRAM NAME HERE
\newcommand{\authname}{Team \#, Team Name
\\ Student 1 name
\\ Student 2 name
\\ Student 3 name
\\ Student 4 name} % AUTHOR NAMES                  

\usepackage{hyperref}
    \hypersetup{colorlinks=true, linkcolor=blue, citecolor=blue, filecolor=blue,
                urlcolor=blue, unicode=false}
    \urlstyle{same}
                                


\begin{document}

\title{Project Title: System Verification and Validation Plan for The Nursery Project} 
\author{Aaron Billones, billonea\\Gillian Ford, fordg\\Juan Moncada, moncadaj\\Steven Ramundi, ramundis}

\date{November 2, 2022}


\maketitle
\thispagestyle{empty}



\pagenumbering{roman}

\begin{tabularx}{\textwidth}{p{3cm}p{4cm}X}
    \toprule {\bf Date} & {\bf Version} & {\bf Notes}\\
    \midrule
    2022-11-02 & Juan Moncada,& Initial release\\&Aaron Billones,\\&Steven Ramundi,\\&Gillian Ford \\
    
    \bottomrule
\end{tabularx}

\newpage

\tableofcontents

\listoftables

\newpage

\section{Symbols, Abbreviations and Acronyms}

\renewcommand{\arraystretch}{1.2}
\begin{tabular}{l l} 
  \toprule		
  \textbf{symbol} & \textbf{description}\\
  \midrule 
  CR & Conveyor Functional Requirement\\
  NFR & Non-Functional Requirement\\
  PDR & Pot Dispensing Functional Requirement\\
  TDR & Tray Dispensing Functional Requirement\\
  VR & Verification Functional Requirement\\
  SRS & Software Requirements Specification\\
  TDST & Tray Dispenser Subsystem Test\\
  PDST & Pot Dispenser Subsystem Test\\
  CST & Conveyor Subsystem Test\\
  VST & Verification Subsystem Test\\
  SCT & Safety Critical Test\\
  PT & Precision Test\\
  RT & Reliability Test\\
  EPET & Expected Physical Environment Test\\
  LED & Light Emitting Diode\\
  LCD & Liquid-Crystal Display\\
  SRT & Speed Requirements Test\\
  ART & Accessibility Requirements Test\\
  LRT & Learning Requirements Test\\

  \bottomrule
\end{tabular}\\



\newpage

\pagenumbering{arabic}

This document ... \wss{provide an introductory blurb and roadmap of the
  Verification and Validation plan}

\section{General Information}

\subsection{Summary}

\wss{Say what software is being tested.  Give its name and a brief overview of
  its general functions.}

\subsection{Objectives}

\wss{State what is intended to be accomplished.  The objective will be around
  the qualities that are most important for your project.  You might have
  something like: ``build confidence in the software correctness,''
  ``demonstrate adequate usability.'' etc.  You won't list all of the qualities,
  just those that are most important.}

\subsection{Relevant Documentation}

\wss{Reference relevant documentation.  This will definitely include your SRS
  and your other project documents (design documents, like MG, MIS, etc).  You
  can include these even before they are written, since by the time the project
  is done, they will be written.}

\citet{SRS}

\section{Plan}

\wss{Introduce this section.   You can provide a roadmap of the sections to
  come.}

\subsection{Verification and Validation Team}

\wss{Your teammates.  Maybe your supervisor.
  You shoud do more than list names.  You should say what each person's role is
  for the project's verification.  A table is a good way to summarize this information.}

\subsection{SRS Verification Plan}

\wss{List any approaches you intend to use for SRS verification.  This may include
  ad hoc feedback from reviewers, like your classmates, or you may plan for 
  something more rigorous/systematic.}

\wss{Maybe create an SRS checklist?}

\subsection{Design Verification Plan}

\wss{Plans for design verification}

\wss{The review will include reviews by your classmates}

\wss{Create a checklists?}

\subsection{Verification and Validation Plan Verification Plan}

\wss{The verification and validation plan is an artifact that should also be verified.}

\wss{The review will include reviews by your classmates}

\wss{Create a checklists?}

\subsection{Implementation Verification Plan}

\wss{You should at least point to the tests listed in this document and the unit
  testing plan.}

\wss{In this section you would also give any details of any plans for static verification of
  the implementation.  Potential techniques include code walkthroughs, code
  inspection, static analyzers, etc.}

\subsection{Automated Testing and Verification Tools}

\wss{What tools are you using for automated testing.  Likely a unit testing
  framework and maybe a profiling tool, like ValGrind.  Other possible tools
  include a static analyzer, make, continuous integration tools, test coverage
  tools, etc.  Explain your plans for summarizing code coverage metrics.
  Linters are another important class of tools.  For the programming language
  you select, you should look at the available linters.  There may also be tools
  that verify that coding standards have been respected, like flake9 for
  Python.}

\wss{If you have already done this in the development plan, you can point to
that document.}

\wss{The details of this section will likely evolve as you get closer to the
  implementation.}

\subsection{Software Validation Plan}

\wss{If there is any external data that can be used for validation, you should
  point to it here.  If there are no plans for validation, you should state that
  here.}

\wss{You might want to use review sessions with the stakeholder to check that
the requirements document captures the right requirements.  Maybe task based
inspection?}

\wss{This section might reference back to the SRS verification section.}

\section{System Test Description}
	
\subsection{Tests for Functional Requirements}

The following section includes system test cases for functional requirements.
The tests are designed in such a way to ensure that all the functional requirements are met.
For reference of the functional requirements, please review the SRS document.


\subsubsection{Pot-pulator Complete System Testing}

\wss{It would be nice to have a blurb here to explain why the subsections below
  cover the requirements.  References to the SRS would be good here.  If a section
  covers tests for input constraints, you should reference the data constraints
  table in the SRS.}
		
\paragraph{Title for Test}

\begin{enumerate}

\item{test-id1\\}

Control: Manual versus Automatic
					
Initial State: 
					
Input: 
					
Output: \wss{The expected result for the given inputs}

Test Case Derivation: \wss{Justify the expected value given in the Output field}
					
How test will be performed: 
					
\item{test-id2\\}

Control: Manual versus Automatic
					
Initial State: 
					
Input: 
					
Output: \wss{The expected result for the given inputs}

Test Case Derivation: \wss{Justify the expected value given in the Output field}

How test will be performed: 

\end{enumerate}

\subsubsection{Tray Dispenser Subsystem Testing}

\begin{enumerate}
        
  \item{TDST-01: \textbf{Tray Stack Detection}}
  
  Control: Static, Manual
            
  Initial State: No trays present in the stack. Trays present in the stack.
            
  Input: Sensor reads the status of tray stack.
            
  Output: Sends a signal/bit to microprocessor that tells the system there are/aren't trays present.
   
  
  Test Case Derivation: The observed signal/bit is the expected value. The subsystem does not operate when no trays are present.
  
  How test will be performed: All other sensors and subsystems will be switched off. 
  All trays will be removed from the stack. The detection bit will be observed. 
  Then trays will be placed in the stack, and the detection bit will be observed.
\\
  \item{TDST-02: \textbf{Operation from Tray Stack Detection}}
  
  Control: Dynamic, Manual
            
  Initial State: Some amount of trays in the stack.
            
  Input: Sensor reads the status of tray stack.
            
  Output: Subsystem operates or remains idle.
  
  Test Case Derivation: If no trays are present, the subsystem will not operate 
  and remain ready in the idle state. Otherwise, operate normally.
  
  How test will be performed: All other sensors and subsystems will be switched off. 
  Trays will be removed from the stack and operation will be observed. Trays will be 
  put in the stack and operation will be observed.\\

  \item{TDST-03: \textbf{Tray from Stack to Conveyor}}
  
  Control: Dynamic, Manual
            
  Initial State: There is a stack of trays beside the vacant conveyor with the subsystem in idle position.
            
  Input: Stack of trays.
            
  Output: One tray from the stack is placed onto the end of the conveyor and returns to idle position.
  
  Test Case Derivation: There is a tray in the correct designated position.
  The subsystem moves into the ready idle state to retrieve more trays.
            
  How test will be performed: All other sensors and subsystems will be switched off. 
  The system will be manually activated to retrieve one tray from its stack.
  The success or failure will be observed.
\\
  \item{TDST-04: \textbf{Verify Tray Status on Conveyor}}
  
  Control: Dynamic, Manual
            
  Initial State: Tray put on conveyor.
            
  Input: Sensor reads the status of tray on conveyor.
            
  Output: Subsystem continues operation or stops.
  
  Test Case Derivation: Subsystem continues operation (when successful) or stops (when tray is stuck/fails to move on conveyor).
  
  How test will be performed: Trays will be fed onto the conveyor correctly. Results will be observed.
  then trays will be placed stuck on purpose. Results will be observed.
  
  \end{enumerate}

\subsubsection{Pot Dispenser Subsystem Testing}

\begin{enumerate}
 
  \item{PDST-01: \textbf{Pot from Stack to Tray}}
  
  Control: Dynamic, Manual
            
  Initial State: Pot dispenser loaded with two pots
            
  Input: Simulated sensor input, two pot locations of tray directly below pot dispenser
            
  Output: Pot dispenser will dispense two pots into designated pot locations on tray
  
  Test Case Derivation: Pot dispenser will dispense pots into correctly positioned tray as it is prompted to
            
  How test will be performed: Tray will be manually placed directly below pot dispenser with pot locations 
  directly below pot stack. Machine will be turned on. Once pots are dispensed, pot dispenser will 
  queue next two pots and tray will be removed.\\

  \item{PDST-02: \textbf{Tray Sensing}}
  
  Control: Dynamic, Manual
            
  Initial State: Mounted sensor with no object being sensed
            
  Input: Manual placement of trays in front of sensor
            
  Output: Sensor will output a signal when the presence of a tray is sensed
  
  Test Case Derivation: Sensor will recognize that a tray is beneath the pot dispenser
            
  How test will be performed: Tray will be manually placed directly in front of the mounted sensor. Signal 
  output from sensor will be analyzed to determine sensor is aware of tray presence. Tray will then be 
  moved forward and output from sensor will be analyzed to confirm sensor is aware that tray is moving. 
  Tray will then be moved forward out of view of sensor and output from sensor will be analyzed to confirm 
  snesor is aware that tray is no longer present.\\

  \item{PDST-03: \textbf{Ability to Dispense 4" Diameter Pots}}
  
  Control: Static, Manual

  How test will be performed: All specifications of pot dispenser will ensure that a 4" diameter pot is able 
  to be dispensed. Measurements and reviews will be conducted by another member of the group any time a change 
  is made to the dispenser during design and build phases. During build phase, test will be conducted on both 
  pot dispensers.\\

  \item{PDST-04: \textbf{Ability to Store/Sispense Multiple Pots}}
  
  Control: Dynamic, Manual

  Initial State: Pot dispenser loaded with pots

  Input: Ten pots, simulated sensor input

  Output: Pot dispenser will dispense two pots, reload with two pots from stack, dispense two pots, etc. until
  pot storage is empty

  Test Case Derivation: Pot dispenser will complete 5 cycles of dispensing, storing and dispensing 10 pots in total

  How test will be performed: Pot dispenser will be loaded with 10 pots, 5 per side. Sensor input will be simulated
  to indicate presence of tray. Pot dispenser will complete 5 cycles of dispensing, at which point pot storage will
  be spent.\\
  
  \item{PDST-05: \textbf{Pot Storage Detection}}
  
  Control: Dynamic, Manual

  Initial State: Pot dispenser with no pots in storage

  Input: N/A

  Output: Pot storage sensor will output a signal when no trays are detected in pot storage

  Test Case Derivation: Sensor will recognize that no pots are sensed in pot storage

  How test will be performed: All pots will be removed from pot storage. Signal output from sensor will be
  analyzed to confirm sensor is aware that pot storage is empty.\\

\end{enumerate}

\subsubsection{Conveyor Subsystem Testing}

\begin{enumerate}
\item{CST-01: \textbf{Conveyor Ability to Move Trays}}

Control: Dynamic, Manual

Initial State: Conveyor with tray placed at start

Input: Simulated inputs indicating conveyor can start

Output: Constant speed of conveyor motor and belt

Test Case Derivation: Conveyor will recognize tray is present on belt and able to move forward

How test will be performed: A single tray will be placed at the start point on the conveyor belt. The conveyor 
will receive signals indicating that there are no issues with any other subsystems and the tray can be moved 
forward. Behaviour of conveyor will be observed to confirm conveyor has moved tray from start to end with no 
stopping. Test will be interrupted if tray is unable to move forward due to physical interferance or if conveyor stops.
\\
\item{CST-02: \textbf{Conveyor Ability to Stop}}

Control: Dynamic, Manual

Initial State: Conveyor moving tray along belt

Input: Simulated signals from pot dispenser indicating tray is beneath pot dispenser

Output: Conveyor motor and belt come to a stop

Test Case Derivation: Conveyor will receive signal from pot dispenser, indicating the tray is beneath the pot
dispenser, and stop movement of tray

How test will be performed: A single tray will be placed on the conveyor while conveyor is moving. A signal will be 
sent to the conveyor, simulating a signal from the pot dispenser sensor which indicates that the tray is beneath 
the pot dispenser. Behaviour of conveyor will be observed o confirm conveyor brings tray to a stop when signal is 
recognized.\\

\item{CST-03: \textbf{Conveyor Belt Friction}}

Control: Static, Manual

Input: Mass of tray, tilt angle of conveyor belt

Output: Maximum acceleration of conveyor belt

Test Case Derivation: Maximum acceleration based on friction between conveyor belt and tray will be calculated 
and set acceleration/decceleration values will be determined

How test will be performed: 6 trays will be weighed and the mean mass will be calculated. Each tray will be placed on 
the conveyor belt one by one. For each tray, the conveyor belt will be tilted until the tray begins to slip, at which
point the angle at which the belt is tilted will be recorded. The mean of these 6 angles will be calculated. These values
will then be used to approximately determine the maximum acceleration the trays can undergo without slipping, and the
acceleration of the conveyor motor will be set to not exceed 70\% of this value.
\\
\end{enumerate}

\subsubsection{Verification Subsystem Testing}

\begin{enumerate}
  \item{VST-01: \textbf{Verify Correct Number of Pots in Tray}}
  
  Control: Dynamic, Manual
            
  Initial State: One tray filled with some pots placed on the conveyor.
            
  Input: Tray filled with a number of pots.
            
  Output: Returns a count of the number of pots in the tray.
  
  Test Case Derivation: The count read by the subsystem matches the actual number
  of pots in the tray.
            
  How test will be performed: All other sensors and subsystems will be switched off. 
  The subsystem will be manually activated to count the number of pots in the given tray as it moves on the conveyor.
  The success or failure will be observed.\\

  \item{VST-02: \textbf{Operation from Verification Status}}
  
  Control: Dynamic, Manual

  Initial State: Tray has completed counting the number of pots in the tray and deemed it success or fail.

  Input: Status bit for success or fail of the pot verification step.

  Output: Signal to tell the system to continue/stop operation based on status bit.

  Test Case Derivation: The subsystem should signal the main processsor to turn off other subsystems
  when there is a problem in verifying the number of pots (ie. $ actual \neq target $).

  How test will be performed: All other sensors and subsystems will be switched off. 
  The subsystem will be manually activated to count the number of pots in the given tray as it moves on the conveyor.
  The success or failure will send a status bit to the main processor. The status bit will be observed.

\end{enumerate}

\subsection{Tests for Nonfunctional Requirements}

\subsubsection{Safety Critical Testing}

\begin{enumerate}

  \item{SCT-01: \textbf{Tray Dispenser Failure}}

  Type: Dynamic, Manual
					
  Initial State: Tray dispenser functioning normally
					
  Input: Tray dispenser disconnect
					
  Output: System flags tray dispenser failure
					
  How test will be performed: Tray dispenser will be manually disconnected from the system during operation. System response 
  will be analyzed to confirm that a failure in the tray dispenser is recognized.
\\		
  \item{SCT-02: \textbf{Pot Dispenser Failure}}

  Type: Dynamic, Manual
					
  Initial State: Pot dispenser functioning normally
					
  Input: Pot dispenser disconnect
					
  Output: System flags pot dispenser failure
					
  How test will be performed: Pot dispenser will be manually disconnected from the system during operation. System response 
  will be analyzed to confirm that a failure in the pot dispenser is recognized.
\\
  \item{SCT-03: \textbf{Conveyor Failure}}

  Type: Dynamic, Manual
					
  Initial State: Conveyor functioning normally
					
  Input: Conveyor disconnect
					
  Output: System flags conveyor failure
					
  How test will be performed: Conveyor will be manually disconnected from the system during operation. System response 
  will be analyzed to confirm that a failure in the conveyor is recognized.
\\
  \item{SCT-04: \textbf{Verification Failure}}

  Type: Dynamic, Manual
					
  Initial State: Verification functioning normally
					
  Input: Verification disconnect
					
  Output: System flags verification failure
					
  How test will be performed: Verification will be manually disconnected from the system during operation. System response 
  will be analyzed to confirm that a failure in the verification is recognized.


\end{enumerate}

\subsubsection{Precision Testing}

\begin{enumerate}

  \item{PT-01: \textbf{Tray Dispenser Precision}}
  
  Type: Static, Manual

  How test will be performed: Tray storage will be filled to maximum capacity. Tray dispenser will place tray onto conveyor belt. 
  Centre line of tray will be established and measured relative to centre line of conveyor. Test will be repeated for all trays 
  until tray storage is empty. Average offset measurement will be calculated.
\\
  \item{PT-02: \textbf{Pot Dispenser Precision}}

  Type: Static, Manual

  How test will be performed: Pot storage will be filled to 50\% capacity. Pot dispenser will dispense pots into trays. Centred 
  position of each opening in trays will be established and measured relative to centre line of pots. Test will be repeated for all
  pots until pot storage is empty. Average offset measurement will be calculated.
  
\end{enumerate}

\subsubsection{Reliability Testing}

\begin{enumerate}

  \item{RT-01: \textbf{Function Under Vibration}}
  
  Type: Static, Manual

  How test will be performed: Pot-pulator will run continuously for 8 hours, with a tester ensuring pots and trays are available to the 
  machine at all times so operation is never interrupted. Machine will be subject to vibrations resulting from conveyor motor, tray dispenser
  motors, and pot dispenser motors. Behaviour will be observed to ensure machine is able to function under long-term exposure to small amplitude 
  vibration.
\end{enumerate}

\subsubsection{Expected Physical Environment Testing}

\begin{enumerate}

  \item{EPET-01: \textbf{Function Under Aerial Pollution}}
  
  Type: Static, Manual

  How test will be performed: Pot-pulator will run continuously for 8 hours in an environment with an amount of aerial pollution similar to 
  what is expected of the environment at Sheridan Nursery. A tester will be present to ensure that pots and trays are available to the machine
  at all times so operation is never interrupted. Behaviour will be observed to ensure machine is able to function for a long period of time in 
  an environment with aerial pollution.

\end{enumerate}

\subsubsection{Speed Requirements Testing}
\begin{enumerate}
  \item{SRT-01: \textbf{Acceleration Displacement of Trays}}
  
Type: Dynamic, Manual

Initial State: Multiple trays (2-3) on the conveyor.

How test will be performed: The conveyor speed and acceleration will be modified while 
trays are on it. The trays should not show perpendicular axis movement along the conveyor
while in higher than normal speeds.
\\
\item{SRT-02: \textbf{Pot Dispensing Rate}}
  
Type: Dynamic, Automatic

Initial State: Subsystem in idle state ready to dispense pots.

Input: Stack of pots.

Output: Pots dispensing.

Test Case Derivation: The preceding system for filling pots with soil operates at a certain rate which
should be met by the Pot-purlator.

How test will be performed: Pots will be placed in the stack. When turned on, the pots will dispense
at the desired rate (10 pots / 30 sec) and timed using a stopwatch. If the dispenser can meet
these requirements then it passes.
\\
\item{SRT-02: \textbf{Tray Dispensing Rate}}
  
Type: Dynamic, Automatic

Initial State: Subsystem in idle state ready to dispense trays.

Input: Stack of Trays.

Output: Trays dispensing.

Test Case Derivation: The preceding system for filling pots with soil operates at a certain rate which
should be met by the Pot-purlator.

How test will be performed: Trays will be placed in the stack. When turned on, the trays will dispense
at the desired rate (1 tray / 30 sec) and timed using a stopwatch. If the dispenser can meet
these requirements then it passes.

\end{enumerate}


\subsubsection{Learning Requirements Testing}
\begin{enumerate}
  \item{LRT-01: \textbf{Operational Simplicity}}
  
Type: Static				
					
How test will be performed: An individual will be trained on operating the entire system.
After training is complete, the individual must demonstrate that they are able to 
successfully and safely handle all possible scenarios that may occur during operation.
\end{enumerate}

\subsubsection{Accessibility Requirements Testing}
\begin{enumerate}
  \item{ART-01: \textbf{Audio and Visual Indicators}}
  
Type: Static, Manual

Inputs: Trigger signal.

Outputs: A corresponding light, sound, or screen display.

How test will be performed: LEDs, speakers, and LCD screens will be placed in the
system and sent a signal to activate in a specified way.
\end{enumerate}


\subsection{Traceability Between Test Cases and Requirements}
The following table outlines all of the system tests and how they relate to the
relevent requirements. The requirements can be referenced in the SRS document.\\

\begin{longtable}{ |p{4cm}|p{8cm}|  }
  \caption{Corresponding Test IDs and Requirements}
  \label{tab:Table1}\\
  
  \hline
  \textbf{Test ID} & \textbf{Supporting Requirements}\\
  \hline
  TDST-01 &  TDR3, TDR5\\
  \hline
  TDST-02 &  TDR4, TDR5 \\
  \hline
  TDST-03 &  TDR2 \\
  \hline
  TDST-04 &  TDR2 \\
  \hline
  PDST-01 &  PDR2 \\
  \hline
  PDST-02 &  PDR2 \\
  \hline
  PDST-03 &  PDR3 \\
  \hline
  PDST-04 &  PDR4 \\
  \hline
  PDST-05 &  PDR5, PDR6 \\
  \hline
  CST-01 &  CR2 \\
  \hline
  CST-02 &  CR3 \\
  \hline
  CST-03 &  CR4 \\
  \hline
  VST-01 &  VR1 \\
  \hline
  VST-02&  VR2 \\
  \hline
  SCT-01 &  NFR12 \\
  \hline
  SCT-02 &  NFR12 \\
  \hline
  SCT-03 &  NFR12 \\
  \hline
  SCT-04 &  NFR12 \\
  \hline
  PT-01 & NFR13 \\
  \hline
  PT-02 & NFR14 \\
  \hline
  RT-01 & NFR17 \\
  \hline
  EPET-01 & NFR20 \\
  \hline
  LRT-01&  NFR6 \\
  \hline
  ART-01 & NFR7 \\
  \hline
  SRT-01 & NFR8 \\
  \hline
  SRT-02 & NFR9 \\
  \hline
  SRT-03 & NFR10 \\
  \hline
\end{longtable}

\newpage

% \section{Appendix}

% This is where you can place additional information.

% \subsection{Symbolic Parameters}

% The definition of the test cases will call for SYMBOLIC\_CONSTANTS.
% Their values are defined in this section for easy maintenance.

% \subsection{Usability Survey Questions?}

% \wss{This is a section that would be appropriate for some projects.}


\newpage{}
\section*{Appendix --- Reflection}

The information in this section will be used to evaluate the team members on the
graduate attribute of Lifelong Learning.  Please answer the following questions:

\begin{enumerate}
  \item What knowledge and skills will the team collectively need to acquire to
  successfully complete the verification and validation of your project?
  Examples of possible knowledge and skills include dynamic testing knowledge,
  static testing knowledge, specific tool usage etc.  You should look to
  identify at least one item for each team member.

  \begin{enumerate}
    \item{Steven:}\\
    I will acquire skills related to dynamic testing of software and hardware. \\
    \item{Juan:}\\
    This project presents a unique challenge when it comes to group management and teamwork, this being one of few interdisciplinary capstones it is the only 
    capstone project(from my knowledge) that includes groups from vastly different engineering streams. While ECE and CAS Share many common teachings, engineering physics lacks that familiarity.
    this, however, in my opinion, is a great representation of the future. When working in the industry there are very few instances that you will be surrounded by those with similar knowledge sets 
    and backgrounds, most of the time the team you will work with will be of diverce experience and backgrounds. Understanding how to work as a team is detrimental to good productivity. I believe throughout this 
    project I will further develop my project management skills, they will be put to the test as each member of the team will have their expertise and we will be fusing knowledge to create a solution.
    Team management and general coding.\\
    \\
    Throughout this project I also wish to enhance my general coding skill. Which I have always had an interest in coding and have made attempts to use it whenever possible throughout my university career. I am excited to 
    take on a project with a defined goal and no clear solution to improve my coding ability.
    \item{Aaron:}\\
    Communication, team management, and time management are 
    crucial to the success of the project. Working towards improving these skills will
    greatly benefit me and my team members in the future. I also expect to further my knowledge in
    both hardware and software while working on this capstone project so it 
    can propel me into the future of becoming an engineer.\\
    \item {Gillian:}\\
    Throughout this project, I am looking forward to learning skills from the other members of the team. As an Engineering physics student, the areas I would like to improve the most about are software and writing. I would like to be prepared for the industry and potentially working in a job that requires coding because it’s something I enjoy, but I don’t think it’s something I am experienced enough in yet. 
    \noindent In terms of writing and communication, it isn’t often where I am required to write long reports anymore, and I like the challenge provided by creating this extent of documentation. I look forward to my writing skills to improve over the course of this capstone project. 

  \end{enumerate}

  \item For each of the knowledge areas and skills identified in the previous
  question, what are at least two approaches to acquiring the knowledge or
  mastering the skill?  Of the identified approaches, which will each team
  member pursue, and why did they make this choice?

  \begin{enumerate}
    \item{Steven:}\\
    An approach to acquiring dynamic testing skills is practicing simulating testing conditions
    with our microcontrollers. An alternative approach is familiarizing myself with microcontroller 
    documentation and creating testing conditions from the knowledge I obtain. I will pursue the former 
    approach, as I feel the trial-and-error this will enable will be a better approach to learning how
    to effectively test specific portions of the project, while having the added bonus of allowing me to
    learn more about the software and hardware I will be working with.\\
    
    \item{Juan:}\\
        Project management skills can be developed throughout this project in many different ways.
        The first will be in taking a heavy role in the logistical and administrative side of the 
        team, another is to run the project management through an issue tracking system
        like the one provided by git hub. This might be tedious as it adds an extra step to all tasks as they will have to be
        updated and status regularly, but the pros outways the cons as this will help with the overall
        management of the team. Of the two solutions, both will be attempted however more emphasis will 
        be put on the task tracking software as it adds the added benefit of simulating how large
        corporations are running their business.\\
        To improve general coding two approaches can be taken on. the first being 
        that I can take on a software dominant portion and be responsible for its fruition, or I can volunteer
        to take one of these software challenges on with another member in my group who is better versed in
        coding so that we may work hand in hand. This way if there will be support and guidance throughout 
        and could offer a solution faster. The second option, in this case, seems more reasonable as there are
        hard deadlines for this project and the system itself is already ambitious.\\
    \item{Aaron:}\\
    Working towards improving my team communication is something that I strive to do while participating
    in this capstone. This can be achived in several ways. One approach that I will take is 
    to strengthen my bond with the team members. In doing so, I will feel more comfortable in sharing ideas and 
    being constructive when necessary. Another approach is to document what each team member is working on and some of the problems they have run into.
    This will increase my team awareness and ultimately lead to the success of the project. I chose to pursue this skill
    because I know sometimes I want to do a lot of work on my own without developing trust towards my team members. This skill
    will help me when I become an engineer and work with large teams where communication and collaboration is crucial.\\
    \item {Gillian:}\\
    I know I can improve my coding skills effectively by working on something closely to another member of the team who is more skilled in that area. I think this will be very helpful because I can work on a portion of software on my own, and if I have questions or get stuck, I will have someone working on something similar who will understand the problem and can help me troubleshoot. 
    \noindent I can improve writing and communication skills by reviewing and editing not only my own work, but the writing of my peers as well. After each deliverable is finished, I think it is very useful when we go over all the writing as a team and come up with ideas on how each section of the documentation can possibly be improved.

  \end{enumerate}

\end{enumerate}

\end{document}