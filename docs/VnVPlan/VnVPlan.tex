\documentclass[12pt, titlepage]{article}

\usepackage{booktabs}
\usepackage{tabularx}
\usepackage{hyperref}
\usepackage{longtable}
\usepackage{array}
\hypersetup{
    colorlinks,
    citecolor=blue,
    filecolor=black,
    linkcolor=red,
    urlcolor=blue
}
\usepackage[round]{natbib}

%% Comments

\usepackage{color}

\newif\ifcomments\commentstrue %displays comments
%\newif\ifcomments\commentsfalse %so that comments do not display

\ifcomments
\newcommand{\authornote}[3]{\textcolor{#1}{[#3 ---#2]}}
\newcommand{\todo}[1]{\textcolor{red}{[TODO: #1]}}
\else
\newcommand{\authornote}[3]{}
\newcommand{\todo}[1]{}
\fi

\newcommand{\wss}[1]{\authornote{blue}{SS}{#1}} 
\newcommand{\plt}[1]{\authornote{magenta}{TPLT}{#1}} %For explanation of the template
\newcommand{\an}[1]{\authornote{cyan}{Author}{#1}}

%% Common Parts

\newcommand{\progname}{ProgName} % PUT YOUR PROGRAM NAME HERE
\newcommand{\authname}{Team \#, Team Name
\\ Student 1 name
\\ Student 2 name
\\ Student 3 name
\\ Student 4 name} % AUTHOR NAMES                  

\usepackage{hyperref}
    \hypersetup{colorlinks=true, linkcolor=blue, citecolor=blue, filecolor=blue,
                urlcolor=blue, unicode=false}
    \urlstyle{same}
                                


\begin{document}

\title{Project Title: System Verification and Validation Plan for The Nursery Project} 
\author{Aaron Billones, billonea\\Gillian Ford, fordg\\Juan Moncada, moncadaj\\Steven Ramundi, ramundis}

\date{November 2, 2022}


\maketitle
\thispagestyle{empty}



\pagenumbering{roman}

\begin{tabularx}{\textwidth}{p{3cm}p{4cm}X}
    \toprule {\bf Date} & {\bf Version} & {\bf Notes}\\
    \midrule
    2022-11-02 & Juan Moncada,& Initial release\\&Aaron Billones,\\&Steven Ramundi,\\&Gillian Ford \\
    
    \bottomrule
\end{tabularx}

\newpage

\tableofcontents

\listoftables
\wss{Remove this section if it isn't needed}

\listoffigures
\wss{Remove this section if it isn't needed}

\newpage

\section{Symbols, Abbreviations and Acronyms}

\renewcommand{\arraystretch}{1.2}
\begin{tabular}{l l} 
  \toprule		
  \textbf{symbol} & \textbf{description}\\
  \midrule 
  CR & Conveyor Functional Requirement\\
  NFR & Non-Functional Requirement\\
  PDR & Pot Dispensing Functional Requirement\\
  PHC & Physical Constraint\\
  SWC & Software Constraint\\
  TDR & Tray Dispensing Functional Requirement\\
  VR & Verification Functional Requirement\\
  SRS & Software Requirements Specification\\
  TDST & Tray Dispenser Subsystem Test\\
  PDST & Pot Dispenser Subsystem Test\\
  CST & Conveyor Subsystem Test\\
  VST & Verification Subsystem Test\\

  \bottomrule
\end{tabular}\\



\newpage

\pagenumbering{arabic}

This document ... \wss{provide an introductory blurb and roadmap of the
  Verification and Validation plan}

\section{General Information}

\subsection{Summary}

\wss{Say what software is being tested.  Give its name and a brief overview of
  its general functions.}

\subsection{Objectives}

\wss{State what is intended to be accomplished.  The objective will be around
  the qualities that are most important for your project.  You might have
  something like: ``build confidence in the software correctness,''
  ``demonstrate adequate usability.'' etc.  You won't list all of the qualities,
  just those that are most important.}

\subsection{Relevant Documentation}

\wss{Reference relevant documentation.  This will definitely include your SRS
  and your other project documents (design documents, like MG, MIS, etc).  You
  can include these even before they are written, since by the time the project
  is done, they will be written.}

\citet{SRS}

\section{Plan}
The following section will outline the plan for system testing. The testing 
will be divided into 5 categories; the 4 separate subsystems that make up the system and the system.

\subsection{Verification and Validation Team}

The testing will be distributed evenly amongst the members of the team. The interdisciplinary aspect of this 
project means that certain members focused more on certain parts of the capstone, thus verification will be 
done evenly but there will be an emphasis on testing sections in which each member was not a major contributing partner.
The following table shows the main point of focus of each group member and their area of focus on the project.
\begin{center}
  \begin{tabular}{ |l|l|l| } 
    \hline
    \textbf{Team member} & \textbf{Area of Focus } & \textbf{Verification Area of Focus} \\ 
    \hline
    Gillian Ford   & Pot dropping and Verification & Conveyor and tray dispensing \\
    Aaron Billones & Conveyor and tray dispensing & Pot dropping and Verification \\
    Steven Ramundi & Conveyor and tray dispensing & Pot dropping and Verification \\
    Juan Moncada   & Pot dropping and Verification & Conveyor and tray dispensing\\
    \hline
  \end{tabular}
\end{center}

\subsection{Milestones}
The following are milestones for the corresponding testing including the date of expected completion. 
\begin{center}
  \begin{tabular}{ |l|l| } 
    \hline
    \textbf{Testing} & \textbf{expected date of completion } \\ 
    \hline
    Pot dropping Subsystem    & December 5th 2022 \\
    Tray dispensing Subsystem & December 5th 2022 \\
    Conveyor Subsystem        & December 5th 2022 \\
    Verification Subsystem    & January 10th 2023 \\
    Whole System              & January 21th 2023 \\

    \hline
  \end{tabular}
\end{center}

\subsection{SRS Verification Plan}

The design verification plan will be comprised of the verification of 4 of the core subsystems.
 Each of these subsystems will be tested individually to include all possible scenarios, once 
 individual testing has been completed the system will be put together and tested as a whole.\\

The verification of system will take place alongside the end user. Sheridan Nurseries has been
 scheduled to have an early look at the unfinished system so that all functional and non-functional 
 requirements may be looked over and made sure to be fulfilled. This process will also allow for the 
 foundations of new requirements that could be added before the final build of the system. There will 
 also be multiple review sessions including by group members and peers.\\

The review will go over the following checklist to make sure that the system is meeting all functional 
and non-functional requirement. The following is the SRS validation checklist.

\begin{center}
  \begin{tabular}{ |l|l| } 
    \hline
    \textbf{Requirement Validate} & \textbf{Pass or fail } \\ 
    \hline
    System interface should be clear, legible, and obvious. & \\
    Subsystems are communicating with integration board.  & \\
    System is running smoothly, without the presence or audible or visual errors.& \\
    There are no lose or exposed wires or electronics.& \\
    There are no expose gears, chains or moving parts.& \\
    
    \hline
  \end{tabular}
\end{center}


\wss{List any approaches you intend to use for SRS verification.  This may include
  ad hoc feedback from reviewers, like your classmates, or you may plan for 
  something more rigorous/systematic.}

\wss{Maybe create an SRS checklist?}

\subsection{Design Verification Plan}
Each of the subsystem will be tested individually. The following section will outline the individual testing
 for each individual testing subsystem. The following are the design validation checklists for the 

 \subsubsection{Conveyor}
 \begin{center}
  \begin{tabular}{ |m{13cm}|m{3.2cm}| } 
    \hline
    \textbf{Requirement Validate} & \textbf{Pass or fail } \\ 
    \hline
    Conveyor is stable so that movements from subsystems do not cause any vibration, tipping or other unwanted movement. & \\

    \hline
  \end{tabular}
\end{center}

\subsubsection{Tray Dispenser}
\begin{center}
 \begin{tabular}{ |m{13cm}|m{3.2cm}| } 
   \hline
   \textbf{Requirement Validate} & \textbf{Pass or fail } \\ 
   \hline
   Tray dispenser runs smoothly and is not translating any unwanted motion into the system. & \\

   \hline
 \end{tabular}
\end{center}

\subsubsection{Pot Dispenser}
\begin{center}
 \begin{tabular}{ |m{13cm}|m{3.2cm}| } 
   \hline
   \textbf{Requirement Validate} & \textbf{Pass or fail } \\ 
   \hline
   pot dispenser runs smoothly and is not translating any unwanted motion into the system. & \\

   \hline
 \end{tabular}
\end{center}

\subsubsection{Verification}
\begin{center}
 \begin{tabular}{ |m{13cm}|m{3.2cm}| } 
   \hline
   \textbf{Requirement Validate} & \textbf{Pass or fail } \\ 
   \hline
    & \\

   \hline
 \end{tabular}
\end{center}


\wss{Plans for design verification}

\wss{The review will include reviews by your classmates}

\wss{Create a checklists?}

\subsection{Verification and Validation Plan Verification Plan}
As the project progresses there will be the requirement to add, remove or alter verification and 
Validation steps/procedures. In order to capture this as the system progresses in its completion 
when major verification and validation milestones are hit, they will be accompanied with the
review of the verification and validation plan. This review will trigger any updates that will
need to be put through.

\wss{The verification and validation plan is an artifact that should also be verified.}

\wss{The review will include reviews by your classmates}

\wss{Create a checklists?}

\subsection{Implementation Verification Plan}

\wss{You should at least point to the tests listed in this document and the unit
  testing plan.}

\wss{In this section you would also give any details of any plans for static verification of
  the implementation.  Potential techniques include code walkthroughs, code
  inspection, static analyzers, etc.}

\subsection{Automated Testing and Verification Tools}

\wss{What tools are you using for automated testing.  Likely a unit testing
  framework and maybe a profiling tool, like ValGrind.  Other possible tools
  include a static analyzer, make, continuous integration tools, test coverage
  tools, etc.  Explain your plans for summarizing code coverage metrics.
  Linters are another important class of tools.  For the programming language
  you select, you should look at the available linters.  There may also be tools
  that verify that coding standards have been respected, like flake9 for
  Python.}

\wss{If you have already done this in the development plan, you can point to
that document.}

\wss{The details of this section will likely evolve as you get closer to the
  implementation.}

\section{System Test Description}
	
\subsection{Tests for Functional Requirements}

The following section includes system test cases for functional requirements.
The tests are designed in such a way to ensure that all the functional requirements are met.
For reference of the functional requirements, please review the SRS document.


\subsubsection{Pot-pulator Complete System Testing}

\wss{It would be nice to have a blurb here to explain why the subsections below
  cover the requirements.  References to the SRS would be good here.  If a section
  covers tests for input constraints, you should reference the data constraints
  table in the SRS.}
		
\paragraph{Title for Test}

\begin{enumerate}

\item{test-id1\\}

Control: Manual versus Automatic
					
Initial State: 
					
Input: 
					
Output: \wss{The expected result for the given inputs}

Test Case Derivation: \wss{Justify the expected value given in the Output field}
					
How test will be performed: 
					
\item{test-id2\\}

Control: Manual versus Automatic
					
Initial State: 
					
Input: 
					
Output: \wss{The expected result for the given inputs}

Test Case Derivation: \wss{Justify the expected value given in the Output field}

How test will be performed: 

\end{enumerate}

\subsubsection{Tray Dispenser Subsystem Testing}

\begin{enumerate}
        
  \item{TDST-01: \textbf{Tray Stack Detection}}
  
  Control: Static, Manual
            
  Initial State: No trays present in the stack. Trays present in the stack.
            
  Input: Sensor reads the status of tray stack.
            
  Output: Sends a signal/bit to microprocessor that tells the system there are/aren't trays present.
   
  
  Test Case Derivation: The observed signal/bit is the expected value. The subsystem does not operate when no trays are present.
  
  How test will be performed: All other sensors and subsystems will be switched off. 
  All trays will be removed from the stack. The detection bit will be observed. 
  Then trays will be placed in the stack, and the detection bit will be observed.
\\
  \item{TDST-02: \textbf{Operation from Tray Stack Detection}}
  
  Control: Dynamic, Manual
            
  Initial State: Some amount of trays in the stack.
            
  Input: Sensor reads the status of tray stack.
            
  Output: Subsystem operates or remains idle.
  
  Test Case Derivation: If no trays are present, the subsystem will not operate 
  and remain ready in the idle state. Otherwise, operate normally.
  
  How test will be performed: All other sensors and subsystems will be switched off. 
  Trays will be removed from the stack and operation will be observed. Trays will be 
  put in the stack and operation will be observed.\\

  \item{TDST-03: \textbf{Tray from Stack to Conveyor}}
  
  Control: Dynamic, Manual
            
  Initial State: There is a stack of trays beside the vacant conveyor with the subsystem in idle position.
            
  Input: Stack of trays.
            
  Output: One tray from the stack is placed onto the end of the conveyor and returns to idle position.
  
  Test Case Derivation: There is a tray in the correct designated position.
  The subsystem moves into the ready idle state to retrieve more trays.
            
  How test will be performed: All other sensors and subsystems will be switched off. 
  The system will be manually activated to retrieve one tray from its stack.
  The success or failure will be observed.
\\
  \item{TDST-04: \textbf{Verify Tray Status on Conveyor}}
  
  Control: Dynamic, Manual
            
  Initial State: Tray put on conveyor.
            
  Input: Sensor reads the status of tray on conveyor.
            
  Output: Subsystem continues operation or stops.
  
  Test Case Derivation: Subsystem continues operation (when successful) or stops (when tray is stuck/fails to move on conveyor).
  
  How test will be performed: Trays will be fed onto the conveyor correctly. Results will be observed.
  then trays will be placed stuck on purpose. Results will be observed.
  
  \end{enumerate}

\subsubsection{Pot Dispenser Subsystem Testing}

\begin{enumerate}
 
  \item{PDST-01: \textbf{Pot from Stack to Tray}\\}
  
  Control: Dynamic, Manual
            
  Initial State: Pot dispenser loaded with two pots
            
  Input: Simulated sensor input, two pot locations of tray directly below pot dispenser
            
  Output: Pot dispenser will dispense two pots into designated pot locations on tray
  
  Test Case Derivation: Pot dispenser will dispense pots into correctly positioned tray as it is prompted to
            
  How test will be performed: Tray will be manually placed directly below pot dispenser with pot locations 
  directly below pot stack. Machine will be turned on. Once pots are dispensed, pot dispenser will 
  queue next two pots and tray will be removed.\\

  \item{PDST-02: \textbf{Tray Sensing}\\}
  
  Control: Dynamic, Manual
            
  Initial State: Mounted sensor with no object being sensed
            
  Input: Manual placement of trays in front of sensor
            
  Output: Sensor will output a signal when the presence of a tray is sensed
  
  Test Case Derivation: Sensor will recognize that a tray is beneath the pot dispenser
            
  How test will be performed: Tray will be manually placed directly in front of the mounted sensor. Signal 
  output from sensor will be analyzed to determine sensor is aware of tray presence. Tray will then be 
  moved forward and output from sensor will be analyzed to confirm sensor is aware that tray is moving. 
  Tray will then be moved forward out of view of sensor and output from sensor will be analyzed to confirm 
  snesor is aware that tray is no longer present.\\

  \item{PDST-03: \textbf{Ability to Dispense 4" Diameter Pots}\\}
  
  Control: Static, Manual

  Initial State: Pot dispenser mechanism loaded with one pot

  Input: Single pot

  Output: Single pot

  Test Case Derivation: Pot dispenser mechanism will dispense one 4" diameter pot

  How test will be performed: All specifications of pot dispenser will ensure that a 4" diameter pot is able 
  to be dispensed. Measurements and reviews will be conducted by another member of the group any time a change 
  is made to the dispenser during design and build phases. During build phase, test will be conducted on both 
  pot dispensers.\\

  \item{PDST-04: \textbf{Ability to Store/Sispense Multiple Pots}\\}
  
  Control: Dynamic, Manual

  Initial State: Pot dispenser loaded with pots

  Input: Ten pots, simulated sensor input

  Output: Pot dispenser will dispense two pots, reload with two pots from stack, dispense two pots, etc. until
  pot storage is empty

  Test Case Derivation: Pot dispenser will complete 5 cycles of dispensing, storing and dispensing 10 pots in total

  How test will be performed: Pot dispenser will be loaded with 10 pots, 5 per side. Sensor input will be simulated
  to indicate presence of tray. Pot dispenser will complete 5 cycles of dispensing, at which point pot storage will
  be spent.\\
  
  \item{PDST-05: \textbf{Pot Storage Sensing}\\}
  
  Control: Dynamic, Manual

  Initial State: Pot dispenser with no pots in storage

  Input: N/A

  Output: Pot storage sensor will output a signal when no trays are detected in pot storage

  Test Case Derivation: Sensor will recognize that no pots are sensed in pot storage

  How test will be performed: All pots will be removed from pot storage. Signal output from sensor will be
  analyzed to confirm sensor is aware that pot storage is empty.\\

\end{enumerate}

\subsubsection{Conveyor Subsystem Testing}

\begin{enumerate}
\item{CST-01: \textbf{Conveyor Ability to Move Trays}\\}

Control: Dynamic, Manual

Initial State: Conveyor with tray placed at start

Input: Simulated inputs indicating conveyor can start

Output: Constant speed of conveyor motor and belt

Test Case Derivation: Conveyor will recognize tray is present on belt and able to move forward

How test will be performed: A single tray will be placed at the start point on the conveyor belt. The conveyor 
will receive signals indicating that there are no issues with any other subsystems and the tray can be moved 
forward. Behaviour of conveyor will be observed to confirm conveyor has moved tray from start to end with no 
stopping. Test will be interrupted if tray is unable to move forward due to physical interferance or if conveyor stops.

\item{CST-02: \textbf{Conveyor Ability to Stop}\\}

Control: Dynamic, Manual

Initial State: Conveyor moving tray along belt

Input: Simulated signals from pot dispenser indicating tray is beneath pot dispenser

Output: Conveyor motor and belt come to a stop

Test Case Derivation: Conveyor will receive signal from pot dispenser, indicating the tray is beneath the pot
dispenser, and stop movement of tray

How test will be performed: A single tray will be placed on the conveyor while conveyor is moving. A signal will be 
sent to the conveyor, simulating a signal from the pot dispenser sensor which indicates that the tray is beneath 
the pot dispenser. Behaviour of conveyor will be observed o confirm conveyor brings tray to a stop when signal is 
recognized.

\item{CST-03: \textbf{Conveyor Belt Friction}\\}

Control: Static, Manual

Initial State: Conveyor belt

Input: Mass of tray, tilt angle of conveyor belt

Output: Maximum acceleration of conveyor belt

Test Case Derivation: Maximum acceleration based on friction between conveyor belt and tray will be calculated 
and set acceleration/decceleration values will be determined

How test will be performed: 6 trays will be weighed and the mean mass will be calculated. Each tray will be placed on 
the conveyor belt one by one. For each tray, the conveyor belt will be tilted until the tray begins to slip, at which
point the angle at which the belt is tilted will be recorded. The mean of these 6 angles will be calculated. These values
will then be used to approximately determine the maximum acceleration the trays can undergo without slipping, and the
acceleration of the conveyor motor will be set to not exceed 70\% of this value.
\\
\end{enumerate}

\subsubsection{Verification Subsystem Testing}

\begin{enumerate}
  \item{VST-01: \textbf{Verify Correct Number of Pots in Tray}}
  
  Control: Dynamic, Manual
            
  Initial State: One tray filled with some pots placed on the conveyor.
            
  Input: Tray filled with a number of pots.
            
  Output: Returns a count of the number of pots in the tray.
  
  Test Case Derivation: The count read by the subsystem matches the actual number
  of pots in the tray.
            
  How test will be performed: All other sensors and subsystems will be switched off. 
  The subsystem will be manually activated to count the number of pots in the given tray as it moves on the conveyor.
  The success or failure will be observed.\\

  \item{VST-02: \textbf{Operation from Verification Status}}
  
  Control: Dynamic, Manual

  Initial State: Tray has completed counting the number of pots in the tray and deemed it success or fail.

  Input: Status bit for success or fail of the pot verification step.

  Output: Signal to tell the system to continue/stop operation based on status bit.

  Test Case Derivation: The subsystem should signal the main processsor to turn off other subsystems
  when there is a problem in verifying the number of pots (ie. $ actual \neq target $).

  How test will be performed: All other sensors and subsystems will be switched off. 
  The subsystem will be manually activated to count the number of pots in the given tray as it moves on the conveyor.
  The success or failure will send a status bit to the main processor. The status bit will be observed.

\end{enumerate}

\subsection{Tests for Nonfunctional Requirements}

\wss{The nonfunctional requirements for accuracy will likely just reference the
  appropriate functional tests from above.  The test cases should mention
  reporting the relative error for these tests.  Not all projects will
  necessarily have nonfunctional requirements related to accuracy}

\wss{Tests related to usability could include conducting a usability test and
  survey.  The survey will be in the Appendix.}

\wss{Static tests, review, inspections, and walkthroughs, will not follow the
format for the tests given below.}

\subsubsection{Area of Testing1}
		
\paragraph{Title for Test}

\begin{enumerate}

\item{test-id1\\}

Type: Functional, Dynamic, Manual, Static etc.
					
Initial State: 
					
Input/Condition: 
					
Output/Result: 
					
How test will be performed: 
					
\item{test-id2\\}

Type: Functional, Dynamic, Manual, Static etc.
					
Initial State: 
					
Input: 
					
Output: 
					
How test will be performed: 

\end{enumerate}

\subsubsection{Area of Testing2}

...

\subsection{Traceability Between Test Cases and Requirements}
The following table outlines all of the system tests and how they relate to the
relevent requirements. The requirements can be referenced in the SRS document.\\

  
\begin{longtable}{ |p{4cm}|p{8cm}|  }
  \caption{Corresponding Test IDs and Requirements}
  \label{tab:Table1}\\
  
  \hline
  \textbf{Test ID} & \textbf{Supporting Requirements}\\
  \hline
  TDST-01 &  TDR3, TDR5\\
  \hline
  TDST-02 &  TDR4, TDR5 \\
  \hline
  TDST-03 &  TDR2 \\
  \hline
  TDST-04 &  TDR2 \\
  \hline
  PDST-01 &  PDR2 \\
  \hline
  PDST-02 &  PDR2 \\
  \hline
  PDST-03 &  PDR3 \\
  \hline
  PDST-04 &  PDR4 \\
  \hline
  PDST-05 &  PDR5, PDR6 \\
  \hline
  CST-01 &  CR2 \\
  \hline
  CST-02 &  CR3 \\
  \hline
  CST-03 &  CR4 \\
  \hline
  VST-01 &  VR1 \\
  \hline
  VST-02&  VR2 \\
  \hline
\end{longtable}

\newpage

% \section{Appendix}

% This is where you can place additional information.

% \subsection{Symbolic Parameters}

% The definition of the test cases will call for SYMBOLIC\_CONSTANTS.
% Their values are defined in this section for easy maintenance.

% \subsection{Usability Survey Questions?}

% \wss{This is a section that would be appropriate for some projects.}


\newpage{}
\section*{Appendix --- Reflection}

The information in this section will be used to evaluate the team members on the
graduate attribute of Lifelong Learning.  Please answer the following questions:

\begin{enumerate}
  \item What knowledge and skills will the team collectively need to acquire to
  successfully complete the verification and validation of your project?
  Examples of possible knowledge and skills include dynamic testing knowledge,
  static testing knowledge, specific tool usage etc.  You should look to
  identify at least one item for each team member.
  \item For each of the knowledge areas and skills identified in the previous
  question, what are at least two approaches to acquiring the knowledge or
  mastering the skill?  Of the identified approaches, which will each team
  member pursue, and why did they make this choice?
\end{enumerate}

\end{document}