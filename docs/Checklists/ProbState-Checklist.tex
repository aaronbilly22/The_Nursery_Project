\documentclass[12pt]{article}

\usepackage{hyperref}
\hypersetup{colorlinks=true,
    linkcolor=blue,
    citecolor=blue,
    filecolor=blue,
    urlcolor=blue,
    unicode=false}
\urlstyle{same}

\usepackage{enumitem, amssymb}
\newlist{todolist}{itemize}{2}
\setlist[todolist]{label=$\square$}
\usepackage{pifont}
\newcommand{\cmark}{\ding{51}}%
\newcommand{\xmark}{\ding{55}}%
\newcommand{\done}{\rlap{$\square$}{\raisebox{2pt}{\large\hspace{1pt}\cmark}}%
\hspace{-2.5pt}}
\newcommand{\wontfix}{\rlap{$\square$}{\large\hspace{1pt}\xmark}}

\begin{document}

\title{Problem Statement and Goals Checklist}
\author{Spencer Smith}
\date{\today}

\maketitle

% Show an item is done by   \item[\done] Frame the problem
% Show an item will not be fixed by   \item[\wontfix] profit

\begin{itemize}
  
\item The source for the document
  \begin{todolist}
  \item The source is text-based, like \LaTeX{}, or markdown, or emaccs org
  mode, etc.
  \end{todolist}

\item Follows \href{https://github.com/smiths/capTemplate/blob/main/docs/Checklists/Writing-Checklist.pdf} {writing checklist}

\item Overall qualities of documentation
  \begin{todolist}
  \item Scope is clear
  \item Document is abstract (for instance, it does not mention the programming
    language (unless this is a true constraint) or the algorithm
    choices (unless the algorithm is the ``what'' rather than the ``how''))
  \item Output of software is unambiguous
  \item Inputs to software are unambiguous
  \item Document is simple (for instance, there is no need for a table of
    contents)
  \item Stakeholders are clearly identified
  \item Environment is clearly identified
  \item Importance of the problem is convincingly stated
  \item Goals provide selling features of the product
  \item Goals are measurable
\end{todolist}

\end{itemize}

\end{document}
