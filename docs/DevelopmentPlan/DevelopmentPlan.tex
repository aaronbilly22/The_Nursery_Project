\documentclass{article}
\usepackage{hyperref}
\usepackage{booktabs}
\usepackage{tabularx}

\title{The Nursery Project\\ Development Plan}

\author{Aaron Billones billonea\\Gillian Ford, fordg\\Juan Moncada, moncadaj\\Steven Ramundi, ramundis}


\date{September 26, 2022}

%% Comments

\usepackage{color}

\newif\ifcomments\commentstrue %displays comments
%\newif\ifcomments\commentsfalse %so that comments do not display

\ifcomments
\newcommand{\authornote}[3]{\textcolor{#1}{[#3 ---#2]}}
\newcommand{\todo}[1]{\textcolor{red}{[TODO: #1]}}
\else
\newcommand{\authornote}[3]{}
\newcommand{\todo}[1]{}
\fi

\newcommand{\wss}[1]{\authornote{blue}{SS}{#1}} 
\newcommand{\plt}[1]{\authornote{magenta}{TPLT}{#1}} %For explanation of the template
\newcommand{\an}[1]{\authornote{cyan}{Author}{#1}}

%% Common Parts

\newcommand{\progname}{ProgName} % PUT YOUR PROGRAM NAME HERE
\newcommand{\authname}{Team \#, Team Name
\\ Student 1 name
\\ Student 2 name
\\ Student 3 name
\\ Student 4 name} % AUTHOR NAMES                  

\usepackage{hyperref}
    \hypersetup{colorlinks=true, linkcolor=blue, citecolor=blue, filecolor=blue,
                urlcolor=blue, unicode=false}
    \urlstyle{same}
                                


\begin{document}

\maketitle

\begin{table}[hp]
	\caption{Revision History} \label{TblRevisionHistory}
	\begin{tabularx}{\textwidth}{llX}
	\toprule
	\textbf{Date} & \textbf{Developer(s)} & \textbf{Change}\\
	\midrule
	2022-09-25 & Juan Moncada,\\&Aaron Billones,\\&Steven Ramundi,\\&Gillian Ford & Initial release\\
	\bottomrule
	\end{tabularx}
	\end{table}

	\newpage
	
	\section{Stakeholders}
	Sheridan Nurseries Norval Manager
	
	\section{Project Description} 
	The current method of preparing pots and trays to be filled with soil and populated with seeds at Sheridan Nurseries is a process with little to no automation, requiring many manual labour hours. Recently, the owners of these farms have been finding it increasingly more difficult to fill these roles with enough workers to run the operation smoothly, and meet production demands. We aim to aid in this process by designing and implementing a machine that is able to fill trays with pots and prepare them for populating with soil and seeds. This will alleviate the reliance on manual labour and improve the overall efficiency of the farm.
	
	
	\section{Team Meeting Plan}
	
	\section{Team Communication Plan}
	
	\section{Team Member Roles}
	
	\section{Workflow Plan}
	All team members will use the designated public repository on GitHub 
	for inquiries relating to the project. The entirety of the projects’ 
	documentation, task distribution, software/hardware design can be found 
	in \href{https://github.com/aaronbilly22/The_Nursery_Project}{The Nursery Project}.
	
	\subsection{Git Branches}
	The repository presented above consists of three branches: main, develop, and prev-milestone. 
	The workflow within these branches will ensure organized and secure development.
	\begin{itemize}
		\item \emph{main}: This branch is the default branch in the repository. It will contain all working code, all parts/assemblies, and complete documentation. 
		It should always be behind or up-to-date with the \emph{develop} branch.
		\item \emph{branch}: This branch is where all the development will take place.
		 The team will review changes and perform testing in this branch. A pull request 
		 will be submitted and approved by a minimum of two team members prior to being 
		 merged into the \emph{main} branch where the working code should be.
		\item \emph{prev-milestone}: This branch will contain all the code, parts, and documentation 
		of the project’s most recently completed milestone. It will pose as a 
		backup branch that can be referred to at any time for ease of communication and 
		development throughout the team.
	\end{itemize}
	Additional branches may be added when a new feature is being implemented if a 
	team member believes that it will help the workflow of the project.

	\subsection{Git Issues}
	A \href{https://github.com/users/aaronbilly22/projects/1}{Git Project Board} will be used to organize 
	issues and tasks for each team member. The issues may be moved within the project board to the 
	appropriate sections (ToDo, In Progress, Done). Issues will be created upon task assignment and closed 
	when the commit has been merged into the main branch. With the creation of each issue, the appropriate assignee(s), label, project, and milestone is chosen
	to provide the team with useful information about the issue.\\

	\noindent The assignees are the team members that are responsible for completing the respective task.\\

	\noindent The following labels will be the most prominent in the project workflow to accurately describe and classify each issue: 

	\begin{itemize}
		\item Bug – Something isn’t working
		\item Documentation – Additions or improvements to project documentation 
		\item Enhancement – Additions or improvements of hardware/software design features 
		\item Help wanted – Extra attention is needed from another team member 
		\item Question – Further information is requested 
	\end{itemize}
	\noindent The Git project will be selected for the issue to appear on the project board.\\

	\noindent The milestone will be selected to improve organization and timeline awareness.\\
	
	\section{Proof of Concept Demonstration Plan}
	
	The main risk of this project is the size and robustness that the project will have to have. The project deals with the manupulation of tangible objects and thus must be scaled to match this.
	The risk of this project can be minimized by splitting the project to four main sections (conveyor, tray allocator, dropper, and verification) which will should be able to work independantly with standardized inputs and outputs, this way the final project will be 
	the implementations of four systems working together. Proof of concept demonstation will consist of a Cad model of the system as a whole aswell as a physical solution and proof for the main hurdle of each indivdual section.
	
	
	\section{Technology}
	
	\begin{itemize}
	\item C++/Python programming language
	\item PEP8 python linter
	\item CI/CD will not be done as the project consists of four separate systems that will be developed independently by each group member. Tracking errors and bugs will be relatively 
	easy as each group member will be responsible for their own system, and testing will be done before each pull request. Programming responsibilities will only overlap once each 
	system is ready to be integrated into the project. We will still be implementing the measure of pushing code and creating pull requests often.
	\item Solidworks (for parts design)
	\item Stepper motors
	\item Gantry robot
	\item End-effector responsible for dropping
	\item End-effector responsible for grabbing
	\end{itemize}
	
	\section{Coding Standard}
	The project will follow the object-oriented programming paradigm. This will ensure flexibility 
	and control over the complexity of the program while maintaining a level of abstraction.\\

	\noindent Another standard that will be followed is the four-eyes principle. The four-eyes principle 
	will improve team awareness and reduce bugs. This will be achieved in the GitHub repository by 
	working mostly in the \emph{develop} branch and having a team member approve a pull request to be merged into \emph{main}.\\

	\noindent For naming conventions, Pascal Case will be used for naming classes/structures, while camelCase will be used for naming methods/functions.

	\section{Project Scheduling}
	The project will primarily follow the course deliverables found in the 
	\href{https://calendar.google.com/calendar/embed?src=rnboqiaki1k2la7rpu3bn0um58%40group.calendar.google.com&ctz=America%2FToronto}{Google Calendar}. 
	The progress of the project will be tracked on through the relevant milestones using the GitHub Project Board. 
	Ordering parts will also be kept track of to ensure that the project timelines are met.\\

\noindent Task responsibilities will be discussed and organized during weekly meetings to keep up with the deliverables. 
An expected timeframe will be strongly encouraged to adhere to for the tasks assigned to each team member. The project 
will ideally be split equally among all members of the team for accomplishing each deliverable. Throughout the lifecycle 
of the project, members may be subject to a slightly heavier workload than others at specific periods in time, based on 
the strengths of each member depending on the hardware and software deliverable needs. 

\end{document}